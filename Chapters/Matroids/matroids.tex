\documentclass[12pt,oneside]{../../sfsuthesis}  

\RequirePackage{standalone}
\usepackage[draft]{../../MAThesisOutputFormat}
%====================%
% Packages 
%====================%
%\usepackage{accents}  % Better accents. I'm not using this
\usepackage{enumitem} % Better labels
%\usepackage[explicit]{titlesec}
\usepackage[normalem]{ulem}     % Thing
\usepackage{stackrel} % Stack text nicely
\usepackage{xcolor}   % Nicer Colors

%====================%
% Re-Define 
%====================%
% Sort out the true phi
\def \badphi {\phi}
\def \phi {\varphi}

%====================%
% New Commands 
%====================%
%% Nice Letters
% Blackboard Bold
\newcommand{\R}{\mathbb{R}}
\newcommand{\Z}{\mathbb{Z}}
\newcommand{\bbS}{\mathbb{S}}
% Fancy Math Cal Letters
\newcommand{\I}{\mathcal{I}}
\newcommand{\J}{\mathcal{J}}
\newcommand{\cL}{\mathcal{L}}
\newcommand{\cF}{\mathcal{F}}
% The hipster letters
\newcommand{\sM}{\mathsf{M}}
\newcommand{\bSigma}{\boldsymbol{\Sigma}}
\newcommand{\ow}{\overline{w}}
\newcommand{\oP}{\overline{P}}
\newcommand{\oQ}{\overline{Q}}
\newcommand{\ochi}{\overline{\chi}}
%% Math Operators
\newcommand{\Cub}{\operatorname{Cub}}
\newcommand{\oCub}{\overline{\Cub}}
\newcommand{\Vol}{\operatorname{Vol}}
\newcommand{\oVol}{\overline{\Vol}}
\newcommand{\MVol}{\operatorname{MVol}}
%% Math Symbols
\newcommand{\cl}{\mathrm{cl}}
\newcommand{\rk}{\mathrm{rk}}
\newcommand{\Inn}{\mathrm{Inn}}
\newcommand{\cone}{\mathrm{cone}}

%====================%
% Theorem Environs
%====================%
\newtheorem{dummy}{}[section]
\theoremstyle{definition}
\newtheorem*{Result}{Main Result}

%====================%
% Draft Helpers
%====================%
% \todo : Make a box with todo comments
\newcommand{\todo}[1]{\par \noindent
    \framebox{\begin{minipage}[c]{0.95 \textwidth}
            \textcolor{red}{TO DO:}
            #1 \end{minipage}}\par}
\usepackage[backend=biber,style=numeric]{biblatex}
\addbibresource{../../thesis.bib}

\begin{document}


\chapter{Matroids}

Much of our concern is around matroids, so best we get a clear idea of what one is.
\todo{Literally any other introductory sentence to this chapter.}

\section{What is a Matroid?}

Matroids are, in a broad sense, a generalization of the notion of independence among elements of sets.
There are several axiomatizations of matroids, and while all are equivalent proving this is nontrivial enough for the relations to affectionaitly be called \textit{cryptomorphisms}.
We will primarily be concerned with two axiomatizations, one based on the notion of independent sets and another based on what are called flats.


\todo{We should mention the graph and matrix as natural jumping of points into matroids.}
\todo{Introduce a graph and a set of vectors that we can use as the generator of some(/the same?) matroids.
    Then  we can use them as examples as we introduce definitions and whatnot}

\subsection{Independent Set Axioms}

This formalization is both a natural starting point and closest to the original\cite{whitneyAbstractPropertiesLinear1935}.
\begin{definition}[Matroid --- Indegendent Set Axioms]\label{def:MatroidIndpendentAxioms}
    A  \emph{matroid} is an ordered pair \( \sM = (E, \I) \), where \( E \) is a finite set called the \emph{ground set} and \( \I \subseteq 2^E \), with the properties:
    \begin{enumerate}[label=\roman*.]
        \item \( \emptyset \in \I \).
        \item If \( I \in \I \) and \( I' \subseteq I \), then \( I' \in \I \).
        \item If \( I_1, I_2 \in \I \) and \( |I_1| \leq |I_2|\), then there exists some \( e \in I_2 \setminus I_1 \)
              such that \( I_1 \cup \{e\} \in \I \).
    \end{enumerate}
\end{definition}
\todo{Do I cite definitions like this? I could cite Oxley or something}

Those with some background in linear algebra may be able to see how this holds for some finite collection of vectors.
And, if you consider a matrix is just a nice way of holding on to a finite set of vectors, you may begin to see where our matroids got their name.

\todo{Work through our example graph and vectors to turn them into a ground set and independent set.}

\subsection{The Flat Axioms}

The second axiomatization of matroids we will need is, perhaps, slightly less intuitive.
\begin{definition}[Matroid --- Flat Axioms]{\label{def:MatroidFlatAxioms}}
    A \emph{matroid} is an ordered pair \( \sM = (E, \cF) \), where \( E \) is a finite set of elements called the ground set and \( F \subseteq 2^E \) is a collection of \emph{flats}
    such that
    \begin{enumerate}[label=\roman*.]
        \item \( E \in \cF \).
        \item If \(F_1, F_2 \in \cF\), then \( F_1 \cap F_2 \in \cF \)
        \item If \( F \in \cF \) and \(F_1, F_2, \dots , F_k \in \cF \) are the minimal, with respect to inclusion, flats containing \( F \),
              then the sets \( F_1 \setminus F,\, F_2 \setminus F,\, \dots,\, F_k \setminus F \) partition \( E \setminus F \).
    \end{enumerate}
\end{definition}

\begin{definition}[Closure]{\label{def:Closure}}
    \hfill
    \todo{Define closure in terms of flats? Or closure such that flats are closed...}

\end{definition}


\begin{definition}[Rank]{\label{def:rank}}
    Let \( \sM = (E, \I) \) be a matroid and \( X \subseteq E \)
    \todo{If we define closure with respect just to flats then we can define rank based on closure.
        Or if we go the other way around, we can define closure based on rank (and flats based on closure?)}
\end{definition}

\todo{This feels a bit backwards when we start by introducing the independent set axioms first.
    Oxley, for example, first defines a basis of a matroid,
    then the restriction of a matroid to a subset of the ground set,
    then closure,
    then defines rank as the cardinality of the basis of the restriction
    (also there's a proof that bases are all the same size).

    Should I ease the transition between definitions that way? It would take up space, but I have space to take up I suppose.}

\begin{lemma}[The Collection of Flats Form a Lattice]
    \hfill
    \todo{This seems important}
\end{lemma}

\section{Matroids from Graphs; Matroids from Matrices}

\subsection{Graphical Matroids}
\todo{Maybe here we make a matroid from our graph?
    Or we do it in subsubsections as we define matroids}

\subsection{Representable Matroids}
\todo{If we didn't do it above, we do it here}

\subsection{Unrepresentable Matroids}
\todo{Fun fact about how most matroids don't come from a matrix.}

\section{The Characteristic Polynomial}

\subsection{The Chromatic Polynomial of a Graph}
Some motivation.
\todo{Work out the chromatic polynomial of the example graph we've been using}

\subsection{The Characteristic Polynomial of a Matroid}
\todo{Define the Characteristic Polynomial}

\subsubsection{Relation between Characteristic Polynomial (of a Matroid of a Graph) and Chromatic Polynomial (of a Graph)}
\todo{Give the basic little translation}

\subsection{The Reduced Characteristic Polynomial}
\todo{Show what reduced means.
    Segway into Chow Rings. Chapter end?}
\end{document}