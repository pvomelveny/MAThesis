\documentclass[12pt,oneside]{../../sfsuthesis} 
 
\RequirePackage{standalone}
\usepackage[draft]{../../MAThesisOutputFormat}
%====================%
% Packages 
%====================%
%\usepackage{accents}  % Better accents. I'm not using this
\usepackage{enumitem} % Better labels
%\usepackage[explicit]{titlesec}
\usepackage[normalem]{ulem}     % Thing
\usepackage{stackrel} % Stack text nicely
\usepackage{xcolor}   % Nicer Colors

%====================%
% Re-Define 
%====================%
% Sort out the true phi
\def \badphi {\phi}
\def \phi {\varphi}

%====================%
% New Commands 
%====================%
%% Nice Letters
% Blackboard Bold
\newcommand{\R}{\mathbb{R}}
\newcommand{\Z}{\mathbb{Z}}
\newcommand{\bbS}{\mathbb{S}}
% Fancy Math Cal Letters
\newcommand{\I}{\mathcal{I}}
\newcommand{\J}{\mathcal{J}}
\newcommand{\cL}{\mathcal{L}}
\newcommand{\cF}{\mathcal{F}}
% The hipster letters
\newcommand{\sM}{\mathsf{M}}
\newcommand{\bSigma}{\boldsymbol{\Sigma}}
\newcommand{\ow}{\overline{w}}
\newcommand{\oP}{\overline{P}}
\newcommand{\oQ}{\overline{Q}}
\newcommand{\ochi}{\overline{\chi}}
%% Math Operators
\newcommand{\Cub}{\operatorname{Cub}}
\newcommand{\oCub}{\overline{\Cub}}
\newcommand{\Vol}{\operatorname{Vol}}
\newcommand{\oVol}{\overline{\Vol}}
\newcommand{\MVol}{\operatorname{MVol}}
%% Math Symbols
\newcommand{\cl}{\mathrm{cl}}
\newcommand{\rk}{\mathrm{rk}}
\newcommand{\Inn}{\mathrm{Inn}}
\newcommand{\cone}{\mathrm{cone}}

%====================%
% Theorem Environs
%====================%
\newtheorem{dummy}{}[section]
\theoremstyle{definition}
\newtheorem*{Result}{Main Result}

%====================%
% Draft Helpers
%====================%
% \todo : Make a box with todo comments
\newcommand{\todo}[1]{\par \noindent
    \framebox{\begin{minipage}[c]{0.95 \textwidth}
            \textcolor{red}{TO DO:}
            #1 \end{minipage}}\par}
\usepackage[backend=biber,style=alphabetic]{biblatex}
\addbibresource{../../thesis.bib}

\begin{document}

\chapter{Chow Rings}

It is time now to delve into the world of algebra, developing the notion of a Chow ring of a matroid.
The primary goal of this section will be to establish the link between the Chow ring and the characteristic polynomial of a matroid.
This will form the first segment of our bridge from combinatorics to geometry.

This section will take some algebraic knowledge for granted; we are not going to define a ring, for example.
The main takeaway will be in the last section and should be enough to move on to the next chapter.
However, even for those with some basic algebra we don't expect Chow rings to be a familiar object, so our first order of business is to define them.

\section{What is a Chow Ring?}

Broadly, Chow rings are a tool from algebraic geometry for studying the intersections of algebraic varieties.
Chow groups of an algebraic variety are equivalence classes of algebraic cycles of its subvarieties, graded by their codimension.
They were named after Wei-Liang Chow, who formalized these cycles in \cite{chowEquivalenceClassesCycles1956}.
Under certain conditions, a product structure can be induced on these groups to give a ring, which encodes additional information about the intersections of the subvarieties.

For those who don't already have some idea of what a Chow ring is, the above probably invites more questions than it clarifies.
Happily, as hinted by the fact that the title of this chapter is not ``A Brief Introduction to Algebraic Geometry and Intersection Theory'', we will not have to tackle the full theory to get a result here.
While there are algebraic varieties hiding in the wings, we will see that the combinatorial data of a matroid allow us to define its corresponding Chow ring quite directly.

\subsection{The Chow Ring of a Matroid}

For our purposes, we will define the Chow ring to be this ``short-cut'' construction.
That this construction corresponds to the idea of the Chow ring presented above, at least for representable matroids, comes from the work started by De Concini and Procesi~\cite{deconciniWonderfulModelsSubspace1995} and generalized to the form we will use by Feichtner and Yuzvinsky~\cite{feichtnerChowRingsToric2004}.
\begin{definition}[The Chow Ring of a Matroid]\label{def:chowRing}
    Let \( \cM = (E, \cL)\) be a matroid.
    Associate a polynomial ring with \( \cM \) given by
    \[
        P_\cM = \R[x_F \; | \; F \in \cL^\ast],
    \]
    and let
    \begin{align*}
        I_\cM & = \left\langle x_{F_1}x_{F_2} \; \middle| \; F_1 \not\subseteq F_2\, \text{and}\, F_2 \not\subseteq F_1 \right\rangle, \\
        J_\cM & = \left\langle \sum_{e_1 \in F} x_{F} - \sum_{e_2 \in F} x_{F} \; \middle| \; e_1, e_2 \in E \right\rangle
    \end{align*}
    be ideals of \(P_\cM\).
    The \emph{Chow ring of \( \cM \)} is given by the quotient
    \[
        A^{\bullet}(\cM) = \frac{P_{\cM}}{I_{\cM} + J_{\cM}}.
    \]
\end{definition}

By way of some intuition building, the idea here is to create a polynomial ring with variables corresponding to the proper flats of our matroid, then encode the combinatorial relations of the matroid using a quotient.
We could think of the ideal \( I_{\cM} \) as telling us that any monomial involving flats that don't form a flag are removed.
The ideal \( J_{\cM} \) has a less obvious intuition, but the linear forms that generate it give us useful relations that we will use later.
As always, working a small example will help.

\subsubsection{A Small Chow Ring Example}
Recall our ongoing example matroid \( \sM \), whose ground set is \( E = \{ a,b,c,d \} \) and with proper flats
\[
    \cL^\ast = \{ a, b, c, d, ac, bc, cd, abd \}.
\]
Then we have the polynomial ring
\[
    P_{\cM} = \R[x_{a}, x_{b}, x_{c}, x_{d}, x_{ac}, x_{bc}, x_{cd}, x_{abd}];
\]
i.e.\ a real polynomial ring in 8 variables.
Elements of the ideal \( I_{\cM} \) will be any multiple of a monomial containing variables corresponding to non-comparable flats, such as
\[
    x_a x_d \in I_{\cM} \quad \text{and} \quad x_c x_{abd} \in I_{\cM}.
\]
The ideal \( J_{\cM} \) in turn is generated from differences of sums of all variables that contain a particular ground element. For example,
\begin{align*}
    \sum_{a \in F}x_F - \sum_{c \in F}x_F & = x_a + x_{abd} + x_{ac} - x_c - x_{ac} - x_{cd} \\
                                          & = x_a + x_{abd} - x_c - x_{cd} \in J_{\cM}.
\end{align*}

Now, elements in our Chow ring \( A^\bullet(\cM) \) are equivalence classes, as expected for a quotient ring.
We see that \( J_\cM \) gives us relations like
\[
    [x_a] + [x_{abd}] = [x_c] + [x_{cd}],
\]
or, equivalently,
\[
    [x_a] = [x_c] + [x_{cd}] - [x_{abd}].
\]

In continuing to strive for succinct notation, we will drop the square brackets on elements of the Chow ring going forward when not needed to distinguish them.

\section{The Degree Map}

Now that we have an idea of what the Chow ring is, we can introduce the next key idea, the degree map.
However, to get to the degree map we will need the property that Chow rings are graded.

\subsection{This Will Be Graded}
Our goal here is to show that the Chow ring is a graded ring.
Those that feel that this property is obvious can skip to the next section; for everyone else we will provide a quick summary.
\begin{definition}[Graded Ring]
    A ring \( R \) is \emph{graded} if the underlying additive group of \( R \) can be decomposed into a direct sum
    \[
        R = \bigoplus_{i=0}^\infty R_i
    \]
    where each \( R_i \) is an abelian group such that \( R_i R_j \subseteq R_{i+j} \) for all \( i, j \in \Z_{\geq 0}\).
    Elements of \( R_i \) are called \emph{homogeneous of degree \( i \)}.
\end{definition}
From this definition we can see that any polynomial ring, \( P \) is a graded ring
\[
    P =  \bigoplus_{i=0}^\infty P_i
\]
where each \( P_i \) is, naturally, all homogeneous polynomials of degree \( i \).

Additionally, we can define a specific kind of ideal for graded rings, homogeneous ideals.
\begin{definition}[Homogeneous Ideals]
    Let \( R \) be a graded ring. An ideal \( I \) of \( R \) is \emph{homogeneous} if
    \[
        I = \langle a_1, a_2, \dots \rangle
    \]
    where each \( a_i \) is homogeneous, i.e., \( a_i \in R_k \) for some \( k \in \Z_{\geq 0} \).
\end{definition}
Looking back at our definition, we see that the ideal \( I_\cM \) is generated by quadratic monomials, while the generators of \( J_\cM \) are all linear.
Thus, \( I_\cM \) and \( J_\cM \) are homogeneous ideals.

Definitions in place, we can state a small proposition.
\begin{proposition}
    Let \( R \) be a graded ring and \( I \) a homogeneous ideal.
    Then the quotient \( R/I \) is itself a graded ring.
    Specifically,
    \[
        R/I = \bigoplus_{n=0}^\infty R_n/I_n
    \]
    where \( I_n = I \cap R_n \).
\end{proposition}
We would say a proof for this could be found in any algebra textbook, but it appears to always be left as an exercise for the reader; a tradition we see no reason to break.

Since the Chow ring is the quotient of a polynomial ring and homogeneous ideals, we have that \( A^\bullet(\cM) \) is a graded ring.
Even better, we get the following from \cite[Proposition~5.5]{adiprasitoHodgeTheoryCombinatorial2018}.
\begin{proposition}
    Given a matroid \( \cM \), such that \( rk(\cM) = r + 1 \),
    \[
        A^\bullet = \bigoplus_{i=0}^r A^i(\cM),
    \]
    with \( A^k = \{ 0 \} \) for all other \( k > r \).
\end{proposition}
All the non-zero components of \( A^\bullet(\cM) \) correspond to ranks of flats of \( \cM \) and are empty otherwise.

\subsection{The Degree Map}

We needed that the Chow ring is a graded ring as the degree map is defined only on the top graded component of the ring.
\begin{definition}[Degree Map]\label{def:degMap}
    Let \( \cM \) be a matroid of rank \( r + 1 \).
    The \emph{degree map} of \( \cM \) is the linear map
    \[
        \deg_\cM:\; A^r(\cM) \to \Z
    \]
    such that for any complete flag \( \scrF \subseteq \cL \) of \( \cM \),
    \[
        \deg_\cM \left( \prod_{F \in \scrF}x_F \right) = 1.
    \]
\end{definition}

At first glance, it is not obvious that such a map must exist or that if it does that it would be unique.
However, Adiprasito, Huh, and Katz showed in~\cite[Proposition~5.6]{adiprasitoHodgeTheoryCombinatorial2018} that this map exists for every matroid and is uniquely characterized by this definition.
We will briefly return to the degree map in the next chapter to provide a little more context.

\section{The Reduced Characteristic Polynomial}

Right away we are going to have to confess that all of this Chow ring business doesn't actually relate to the characteristic polynomial directly.
We have made a slight misdirection.
Instead, we have to introduce the actual target of our machinations, the reduced characteristic polynomial.
The definition is quite straightforward, we just require one small proposition about the characteristic polynomial.

% \subsection{Digression: The M\"obius Function and the Characteristic Polynomial}
% Recall we mentioned that Rota defined the characteristic polynomial using the M\"obius function, but avoided it for ease of understanding.
% However, using this alternative definition makes our work in this section quite straightforward.
% Forgive us for trading clarity for elegance.

% \begin{definition}[M\"obius Function]\label{def:mobiusFunc}
%     Let \( \cM \) be a matroid and \( \cL \) be its lattice of flats.
%     Then the \emph{M\"obius function} on \( \cM \) is the map
%     \[
%         \mu_\cM \,:\; \cL \times \cL \to \Z
%     \]
%     defined recursively for any \( F, F' \in \cL \) by
%     \[
%         \mu_\cM(F, F') =
%         \begin{dcases}
%             \quad\; 1 \vphantom{\sum}                                    & \text{if}\; F = F'            \\
%             - \sum_{\mathclap{F \subseteq X \subsetneq F'}}\mu_\cM(F, X) & \text{when}\; F \subsetneq F' \\
%             \quad\; 0                                                    & \text{otherwise}.
%         \end{dcases}
%     \]
% \end{definition}
% Truly, of all the definitions we have presented so far, the M\"obius function most deserves some serious contemplation over a cup of coffee.
% We won't go that far afield here, but we can now use it in an equivalent definition of the characteristic polynomial.
% \begin{definition}[Characteristic Polynomial -- via M\"obius functions]
%     Let \( \cM = (E, \cL) \) be a matroid.
%     Then we may write the characteristic polynomial of \( \cM \) as
%     \[
%         \chi_\cM(z) = \sum_{F \in \cL} \mu_\cM(\emptyset, F) z^{\rk(\cM) - \rk(F)}.
%     \]
% \end{definition}
% This definition does make the importance of the lattice structure of the flats much more clear.
% With this we can present, and prove, our fact about the characteristic polynomial.
% \begin{proposition}\th\label{prop:charOfOneIsZero}
%     Let \( \cM = (E, \cL) \) be any matroid, then
%     \[
%         \chi_\cM(1) = 0.
%     \]
% \end{proposition}
% \begin{proof}
%     This follows directly from the definition of the M\"obius function, and a little reindexing.
%     We see that
%     \begin{align*}
%         \chi_\cM(1) & = \sum_{F \in \cL} \mu_\cM(\emptyset, F) 1^{\rk(\cM) - \rk(F)}                                                                                             \\
%                     & =  \sum_{F \in \cL} \mu_\cM(\emptyset, F)                                                                                                                  \\
%                     & = \mu_\cM(\emptyset, E) +  \sum_{\mathclap{\emptyset \subseteq F \subsetneq E}} \mu_\cM(\emptyset, F)                                                      \\
%                     & = -\sum_{\mathclap{\emptyset \subseteq F \subsetneq E}} \mu_\cM(\emptyset, F) + \sum_{\mathclap{\emptyset \subseteq F \subsetneq E}} \mu_\cM(\emptyset, F) \\
%                     & = 0,
%     \end{align*}
%     with the second to last equality coming from the application of the definition.
% \end{proof}
% From this we have the following corollary and main goal of this section.
% \begin{corollary}\th\label{thm:charPolyFactor}
%     For any matroid \( \cM \), the characteristic polynomial \( \chi_\cM(z) \) has a factor of \( (z - 1) \).
% \end{corollary}
% Digression aside, we can get back to defining the reduced characteristic polynomial.

% \subsection{The Reduced Characteristic Polynomial}

\begin{proposition}\th\label{thm:charPolyFactor}
    For any matroid \( \cM \), the characteristic polynomial \( \chi_\cM(z) \) has a factor of \( (z - 1) \).
\end{proposition}
\begin{proof}
    This follows from the recursive definition of the characteristic polynomial we get from \th\ref{thm:charPolyProps}.
    Let \( \cM \) be a matroid.
    If \( \cM \) has a loop then, we have that \( \chi_\cM(z) = 0 \) which trivially has \( (z - 1) \) is a factor.
    Likewise, if \( \cM \) has any element \( e \in E \) such that \( e \) is a coloop, we have that
    \[
        \chi_\cM(z) = (z - 1)\chi_{\cM \setminus e}(z),
    \]
    which we see has \( (z - 1) \) as a factor.

    So, we should assume that \( \cM \) has neither a coloop nor a loop.
    Then we can pick any element \( e \in E \) and use the deletion-contraction property to write
    \[
        \chi_\cM(z) = \chi_{\cM \setminus e}(z) - \chi_{\cM / e}(z),
    \]
    where each matroid in the sum on the right-hand side has a ground set with one fewer element in its ground set than our original matroid.
    If either of these minors have a loop or coloop, that part of the sum has a factor of \( (z - 1) \) and we no longer need to worry about it.
    Otherwise, we can again use the deletion-contraction property to write these as the characteristic polynomial of matroids with one fewer element.
    We can continue this until we have expressed the characteristic polynomial as the sum of characteristic polynomials of matroids that either have a loop, a coloop, or a single element in its ground set.

    Up to isomorphism, there are only two matroids with a single element.
    Both have a ground set \( E = \{a\} \), and they have independent sets \( \I = \{a\} \) or \( \I = \emptyset \).
    If \( a \) is independent, then it is a coloop, since it increases the rank of every subset of \( E \setminus a \).
    If \( a \) is not independent, then it is a loop by definition.

    So this final sum actually consists only of matroids with coloops or loops, and so each has a factor of \( (z - 1) \).
    We conclude that \( \chi_{\cM}(z) \) must have a factor of \( (z - 1) \).
\end{proof}
Now that we are sure there will always be a factor of \( (z - 1) \), the reduced characteristic polynomial is easy to define in terms of the characteristic polynomial.
\begin{definition}[Reduced Characteristic Polynomial]\label{def:reducedCharPoly}
    Let \( \cM \) be a matroid of rank \( r + 1 \).
    The \emph{reduced characteristic polynomial} of \( \cM \) is
    \[
        \ochi_\cM(z) = \frac{\chi_\cM(z)}{(z-1)}.
    \]
    Collecting the powers of \( z \) and writing the reduced characteristic polynomial as
    \[
        \ochi_\cM(z) = \sum_{k=0}^r (-1)^k\ow_k z^{r-k},
    \]
    we define \( \ow_0, \ow_1, \dots, \ow_r \) as the reduced coefficients of \( \ochi_\cM(z) \).
\end{definition}
That this is well-defined for any matroid follows from \th\ref{thm:charPolyFactor}.
The following proposition tells us why these coefficients are going to be important.
\begin{lemma}\th\label{thm:reducedImpliesOriginal}
    Let  \( \ow_0, \ow_1, \dots, \ow_r \) be the reduced coefficients of the reduced characteristic polynomial \( \ochi_\cM(z) \) of some matroid \( \cM \).
    If
    \[
        \{\ow_0, \ow_1, \dots, \ow_r \}
    \]
    is a log-concave sequence, then the Whitney numbers of the first kind of \( \cM \),
    \[
        w_0, w_1, \dots, w_{r+1},
    \]
    also form a log-concave sequence.
\end{lemma}
\begin{proof}
    % First, we would like to recall a readily varifiable fact about log-concave sequences of strictly positive numbers.
    % If \( \{ \alpha_0, \alpha_1, \dots, \alpha_k \} \) is a log-concave sequence if and only if
    % \[
    %     \frac{\alpha_1}{\alpha_0} \geq \frac{\alpha_2}{\alpha_1} \geq \cdots \geq \frac{\alpha_k}{\alpha_{k-1}}.
    % \]
    % This follows immediately from the property that \( \alpha_k^2 \geq \alpha_{k-1}\alpha_{k+1} \).

    % Now on to the main part of our proof.
    Let \( \cM \) be a matroid of rank \( r + 1 \) and
    \[
        \{\ow_0, \ow_1, \dots, \ow_r \}
    \]
    be the reduced coefficients of the reduced characteristic polynomial \( \ochi_\cM(z) \).
    Following from the deletion-contraction property, the reduced characteristic coefficients are positive.
    Assume that \( \{\ow_0, \ow_1, \dots, \ow_r \} \) is a log-concave sequence.

    To recover the characteristic polynomial of \( \cM \) we can multiply our polynomial by a factor of \( (z-1) \).
    This gives us
    \begin{align*}
        (z - 1)\sum_{k=0}^r(-1)^k \ow_k z^{r-k} & = z\sum_{k=0}^r(-1)^k \ow_k z^{r-k} - \sum_{k=0}^r(-1)^k \ow_k z^{r-k}              \\
                                                & = \sum_{k=0}^r(-1)^k \ow_k z^{r-k+1} + \sum_{k=0}^r(-1)^{k+1} \ow_k z^{r-k}         \\
                                                & = \sum_{k=0}^r(-1)^k \ow_k z^{r-k+1} + \sum_{k=1}^{r+1}(-1)^{k} \ow_{k-1} z^{r-k+1} \\
                                                & = \ow_0 z^{r+1} + \sum_{k=1}^r (-1)^k (\ow_k + \ow_{k-1})z^{r-k} + (-1)^{r+1}\ow_r,
    \end{align*}
    where the last two equalities are just a result of some clever reindexing.
    For convenience, we will define \( \ow_{-1} = \ow_{r+1} = 0 \).
    We note that \( \{ \ow_{-1}, \ow_0, \ow_1, \dots, \ow_r, \ow_{r+1} \} \) remains a log-concave sequence.
    This lets us write
    \[
        \ow_0 z^{r+1} + \sum_{k=1}^r (-1)^k (\ow_k + \ow_{k-1})z^{r-k+1} + (-1)^{r+1}\ow_r = \sum_{k=0}^{r+1}(-1)^k(\ow_k + \ow_{k-1})z^{r-k}.
    \]
    Recalling the definition of the characteristic polynomial and its coefficients, we have shown that
    \[
        w_k = \ow_k + \ow_{k-1},
    \]
    for \( 0 \leq k \leq r+1 \).

    Now, we wish to show the log-concavity of the coefficients, so consider the expression \( w_k^2 - w_{k-1}w_{k+1} \).
    If we can show this must be non-negative for any \( 0 < k < r+1 \), we will have that the sequence is log-concave.
    First let's manipulate the expression a bit to get
    \begin{align*}
        w_k^2 - w_{k-1}w_{k+1} & = (\ow_k + \ow_{k-1})^2 - (\ow_{k-1} + \ow_{k-2})(\ow_{k+1} + \ow_{k})                                                                     \\
                               & = (\ow_k^2 + 2\ow_k \ow_{k-1} + \ow_{k-1}^2 \big) - \big( \ow_{k-1} \ow_{k+1} + \ow_{k-1}\ow_{k} + \ow_{k+1}\ow_{k-2} + \ow_{k} \ow_{k-2}) \\
                               & = (\ow_k^2 - \ow_{k-1}\ow_{k+1}) + (\ow_{k-1}^2 - \ow_{k-2}\ow_{k})                                                                        \\
                               & \quad + (\ow_k \ow_{k-1} - \ow_k \ow_{k-1}) + (\ow_k \ow_{k-1} - \ow_{k+1} \ow_{k-2} )                                                     \\
                               & = (\ow_k^2 - \ow_{k-1}\ow_{k+1}) + (\ow_{k-1}^2 - \ow_{k-2}\ow_{k}) + (\ow_k \ow_{k-1} - \ow_{k+1} \ow_{k-2} ).
    \end{align*}
    It is now this expression we want to show is non-negative.
    Immediately we have that
    \begin{align*}
        \ow_k^2 - \ow_{k-1}\ow_{k+1} \geq 0, \\
        \ow_{k-1}^2 - \ow_{k-2}\ow_{k} \geq 0
    \end{align*}
    from the assumption of the log-concavity of the reduced coefficients.
    All that's left then is to show that the term
    \[
        \ow_k \ow_{k-1} - \ow_{k+1} \ow_{k-2} \geq 0.
    \]
    If \( k = 1 \) or \( k = r \) then this is immediately true, as \(\ow_{k+1} \ow_{k-2} = 0\) in either of these cases.
    So, we will go forward and assume that \( 1 < k < r \).
    Now, consider that it must be that
    \[
        \frac{\ow_{k-1}}{\ow_{k-2}} \geq \frac{\ow_k}{\ow_{k-1}} \geq \frac{\ow_{k+1}}{\ow_k}.
    \]
    This follows directly from the definition of log-concavity and that they are all positive.
    This means
    \[
        \frac{\ow_{k-1}}{\ow_{k-2}} \geq \frac{\ow_{k+1}}{\ow_k}
    \]
    and so
    \[
        \ow_k \ow_{k-1} \geq \ow_{k+1} \ow_{k-2}
    \]
    giving us
    \[
        \ow_k \ow_{k-1} - \ow_{k+1} \ow_{k-2} \geq 0
    \]
    as desired.

    Having shown that \( w_k^2 - w_{k-1}w_{k+1} \) for all \( 1 \leq k \leq r \) must be non-negative, we conclude that the sequence \( \{ w_k \}_{k=0}^{r+1} \) is log-concave.
\end{proof}
It is actually this sequence, \(\{\ow_k\}_{k=0}^r \), that Adiprasito, Huh, and Katz spent most of \cite{adiprasitoHodgeTheoryCombinatorial2018} proving is log-concave.
This too is our strategy, so these reduced coefficients really are key players in this story.
But we have still yet to link the characteristic polynomial, reduced or not, to the Chow ring.

\subsection{The Divisors \texorpdfstring{\( \alpha \)}{alpha} and \texorpdfstring{\( \beta \)}{beta}}
Right off the bat, let's learn a little jargon.
\begin{definition}[Divisor]
    A \emph{divisor} of a Chow ring \( A^\bullet(\cM) \) is any linear term, i.e., an element of \( A^1(\cM) \).
\end{definition}
We are working towards introducing a proposition from the Adiprasito, Huh, and Katz paper that links certain divisors to the reduced coefficients, via the degree map.
We will start by introducing these special divisors.
\begin{definition}[\texorpdfstring{\( \alpha \)}{alpha} and \texorpdfstring{\( \beta \)}{beta}]\label{def:alphaBeta}
    Let \( \cM \) be a matroid with ground set \( E \).
    For every element \( e \in E \) we define the divisors
    \[
        \alpha_e = \sum_{e \in F} x_F \quad \text{and} \quad \beta_e = \sum_{e \notin F}x_F.
    \]
\end{definition}
\begin{proposition}
    As elements of \( A^\bullet(\cM) \), \(\alpha_{e} = \alpha_{e'} \) and \(\beta_{e} = \beta_{e'} \) for any \( e, e' \in E \).
    Going forward we may refer simply to \( \alpha \) and \( \beta \), since the class is independent of a choice of ground element.
\end{proposition}
\begin{proof}
    Recall that the ideal \( J \) of \( A^\bullet(\cM) \) gives us the relation
    \[
        \sum_{e \in F} x_F = \sum_{e' \in F} x_F
    \]
    for any \( e, e' \in E \).
    Thus, we have
    \begin{align*}
        \alpha_{e} & = \sum_{e \in F} x_F  \\
                   & = \sum_{e' \in F} x_F \\
                   & = \alpha_{e'}
    \end{align*}
    for any ground elements \( e, e' \).
    The argument follows identically for \( \beta \), once we note
    \[
        \beta_e =  \sum_{e \notin F} x_F = \sum_{F \in \cL^\ast} x_F - \sum_{e \in F} x_F.
    \]
    So, in the Chow ring, any choice of ground element for \( \alpha \) and \( \beta \) is equivalent to any other.
\end{proof}

With this, we may state the key takeaway from this chapter.
There is a link between \( \alpha \) and \( \beta \) and the reduced coefficients provided by~\cite[Proposition~9.5]{adiprasitoHodgeTheoryCombinatorial2018}.
\begin{proposition}\th\label{thm:coeffIsDeg}
    Given any matroid \( \cM \) with reduced coefficients \( \ow_0, \dots, \ow_r \), we have the relationship
    \[
        \ow_k = \deg(\alpha^{r-k}\beta^k).
    \]
    for all \( 0 \leq k \leq r \).
\end{proposition}
Just to confirm our understanding of the degree map, recall it is defined on elements of \( A^r(\cM) \).
Since \( \alpha, \beta \in A^1(\cM)\), we will have \( \alpha^{r-k}\beta^k \in A^r(\cM) \) and so this makes sense as input to the degree map.
An alternative proof of~\th\ref{thm:coeffIsDeg} given by Dastidar and Ross,~\cite[Proposition~3.11]{dastidarMatroidPsiClasses2021}, shows that the right-hand side satisfies the deletion-contraction property.

\th\ref{thm:coeffIsDeg} means that if we can prove
\[
    \deg(\alpha^{r-k}\beta^k)^2 \geq \deg(\alpha^{r-k-1}\beta^{k-1}) \deg(\alpha^{r-k+1}\beta^{k+1})
\]
for \( 0 < k < r \) we will have shown that the reduced coefficients are log-concave and thus so too are the original coefficients.
Here now is where we diverge from the strategy put forth in \cite{adiprasitoHodgeTheoryCombinatorial2018}.
They go on to show this relationship in a very algebraic manner, proving that the Chow ring of a matroid has many desirable properties that eventually yield the desired result.
We, on the other hand, will now move into the world of geometry and find a different way to generate our log-concave sequence.
\end{document}