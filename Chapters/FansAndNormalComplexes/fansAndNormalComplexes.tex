\documentclass[12pt,oneside]{../../sfsuthesis} 
 
\RequirePackage{standalone}
\usepackage[draft]{../../MAThesisOutputFormat}
%====================%
% Packages 
%====================%
%\usepackage{accents}  % Better accents. I'm not using this
\usepackage{enumitem} % Better labels
%\usepackage[explicit]{titlesec}
\usepackage[normalem]{ulem}     % Thing
\usepackage{stackrel} % Stack text nicely
\usepackage{xcolor}   % Nicer Colors

%====================%
% Re-Define 
%====================%
% Sort out the true phi
\def \badphi {\phi}
\def \phi {\varphi}

%====================%
% New Commands 
%====================%
%% Nice Letters
% Blackboard Bold
\newcommand{\R}{\mathbb{R}}
\newcommand{\Z}{\mathbb{Z}}
\newcommand{\bbS}{\mathbb{S}}
% Fancy Math Cal Letters
\newcommand{\I}{\mathcal{I}}
\newcommand{\J}{\mathcal{J}}
\newcommand{\cL}{\mathcal{L}}
\newcommand{\cF}{\mathcal{F}}
% The hipster letters
\newcommand{\sM}{\mathsf{M}}
\newcommand{\bSigma}{\boldsymbol{\Sigma}}
\newcommand{\ow}{\overline{w}}
\newcommand{\oP}{\overline{P}}
\newcommand{\oQ}{\overline{Q}}
\newcommand{\ochi}{\overline{\chi}}
%% Math Operators
\newcommand{\Cub}{\operatorname{Cub}}
\newcommand{\oCub}{\overline{\Cub}}
\newcommand{\Vol}{\operatorname{Vol}}
\newcommand{\oVol}{\overline{\Vol}}
\newcommand{\MVol}{\operatorname{MVol}}
%% Math Symbols
\newcommand{\cl}{\mathrm{cl}}
\newcommand{\rk}{\mathrm{rk}}
\newcommand{\Inn}{\mathrm{Inn}}
\newcommand{\cone}{\mathrm{cone}}

%====================%
% Theorem Environs
%====================%
\newtheorem{dummy}{}[section]
\theoremstyle{definition}
\newtheorem*{Result}{Main Result}

%====================%
% Draft Helpers
%====================%
% \todo : Make a box with todo comments
\newcommand{\todo}[1]{\par \noindent
    \framebox{\begin{minipage}[c]{0.95 \textwidth}
            \textcolor{red}{TO DO:}
            #1 \end{minipage}}\par}
\usepackage[backend=biber,style=alphabetic]{biblatex}
\addbibresource{../../thesis.bib}

\begin{document}

\chapter{Bergman Fans and their Normal Complexes}
As we continue our tour of various branches of mathematics, we arrive at geometry.
The primary goal of this chapter is to develop the final segment of our bridge connecting some geometric object back to the Chow ring, and then showing how we can generate log-concave sequences with these objects.
To get there we will provide a quick primer on polyhedral geometry and a classic theorem of convex geometry that generates log-concave sequences.
Then we'll introduce a geometric object associated to a matroid, the Bergman fan, and show how we can use them to make some new objects called a normal complexes.

\section{A Little Polyhedral Geometry, as a Treat}
Really, the basic building blocks we'll be using are not that weird.
It's geometry, we're going to be using some sort of shapes living in some kind of space.
We must admit, however, that we personally struggle visualizing the higher dimensional objects at play, and so must fall back on formalism.

This section is a short crash course on basic elements of polyhedral geometry.
Our treatment of this topic will often parallel that in Ziegler's ``Lectures on Polytopes'' \cite{zieglerLecturesPolytopes1995}, which we recommend for those who'd like a little more depth than presented here.

\subsection{The Cone Zone}
We are going to be using two fundamental kinds of convex shapes, polytopes and cones.
As a reminder, a convex object is one where if you pick any two points in it, the line connecting those points never leaves the shape.
We can, and will, state this formally.
\begin{definition}[Convexity]\label{def:convex}
    Let \( K \subseteq \R^n \).
    We call \( K \) \emph{convex} if for every \( p, q \in K \), we have
    \[
        [p, q] \subseteq K,
    \]
    where \( [p, q] = \{ \lambda p + (1 - \lambda) q \; | \; 0 \leq \lambda \leq 1 \} \) is the line segment between \( p \) and \( q \).
\end{definition}

\begin{figure}[H]

    \begin{subfigure}[t]{0.5\textwidth}
        \centering
        \begin{tikzpicture}[scale=0.7]
            \draw[thick, fill=SeaGreen, opacity=0.5] (-2,-2) -- (-2,2) -- (2,2) -- (2,-2) -- cycle;

            \node[circle, fill=MidnightBlue, scale=0.5] (A) at (1.2, -1.55) {};
            \node[circle, fill=MidnightBlue, scale=0.5] (B) at (-1.8, 0.2) {};
            \draw[thick, MidnightBlue] (A) -- (B);
        \end{tikzpicture}

    \end{subfigure}
    \begin{subfigure}[t]{0.5\textwidth}
        \centering
        \begin{tikzpicture}[scale=0.7]
            \pgfmathsetmacro{\ct}{3} % distance center to tip
            \pgfmathsetmacro{\cc}{\ct*sin(18)/sin(126)} % distance center to corner (sine rule)
            % star
            \draw[thick, fill=Goldenrod, opacity=0.5] (0,0)
            +(90-0*36:\ct) coordinate(T1)
            foreach[evaluate=\x as \nc using int((\x+1)/2),   % number for corner coordinates
                    evaluate=\x as \nt using int((\x+1)/2+1)] % number for tip coordinates
            \x in {1,3,...,9}{
                    -- +(90-\x*36:\cc) coordinate(C\nc) -- +({90-(\x+1)*36}:\ct) coordinate(T\nt)}
            -- cycle;
            % \node[above=1cm] at (T1) {star only};
            % star
            % \draw[ultra thick] (7,0)
            %     +(90-0*36:\ct) coordinate(T1)
            %     foreach[evaluate=\x as \nc using int((\x+1)/2),   % number for corner coordinates
            %             evaluate=\x as \nt using int((\x+1)/2+1)] % number for tip coordinates
            %         \x in {1,3,...,9}{
            %         -- +(90-\x*36:\cc) coordinate(C\nc) -- +({90-(\x+1)*36}:\ct) coordinate(T\nt)}
            %     -- cycle;
            % % the rest
            % \draw[ultra thick] (C2) -- (C3) -- (C4);
            % \draw[ultra thick,fill=green] (T5) -- (C1) -- (C4) -- cycle;
            % \draw[ultra thick,fill=green] (C4) -- (C1) -- (T2) -- (C2) -- (C4);

            % \node[above=1cm] at (T1) {complete};
            \node[circle, fill=red, scale=0.5] (A) at (-1.2, -1.55) {};
            \node[circle, fill=red, scale=0.5] (B) at (-1.6, 0.5) {};
            \draw[thick, red] (A) -- (B);
        \end{tikzpicture}
    \end{subfigure}
    \caption{Everyone's first pair of convex and non-convex shapes}
\end{figure}

While there are generally a few ways one could define polytope and cone, we will use a definition based on construction using some finite collection of points.
In brief, a polytope is a \emph{convex hull} of finitely many points and a cone is the \emph{conic combination} of finitely many generating vectors.
Let's make this formal.
\begin{definition}[Polytope]\label{def:polytope}
    Let \( P \subseteq \R^n \).
    We say \( P \) is a \emph{polytope} if it is the convex hull of some finite set of points \( x_1, x_2, \dots, x_k \).
    That is to say \( P \) is a polytope if
    \[
        P = \conv(\{x_1, \dots, x_k\}),
    \]
    where
    \[
        \conv(\{x_1, \dots, x_k\}) = \left\{ \lambda_1 x_1 + \lambda_2 x_2 + \cdots + \lambda_k x_k \; \middle| \; \lambda_i \geq 0, \sum_{i=0}^k \lambda_i = 1 \right\}
    \]
    is the convex hull of \( x_1, x_2, \dots, x_k \).
\end{definition}
An astute reader may notice that our shorthand for line segment above, \( [x, y] \), is in fact just \( \conv(\{x,y\}) \).
Towards some intuition, we may think of the convex hull as the smallest convex shape that contains all of its generating points.
In two dimensions we like to think of this as stretching a rubber band around a bunch of points and letting it constrict around them.
\begin{figure}[H]
    \begin{subfigure}[t]{0.3\textwidth}
        \centering
        \begin{tikzpicture}[scale=.72]

            \node[circle, fill=Black, scale=0.7] (A) at (-1, -3) {};
            % \node[circle, fill=Black, scale=0.7] (B) at (-0.5, -1.5) {};
            \node[circle, fill=Black, scale=0.7] (C) at (1, 3) {};
            \draw[ultra thick, Black] (A) -- (C);
        \end{tikzpicture}
        \subcaption*{A 1-dimensional polytope}

        \subcaption*{A 1-dimensional polytope}
    \end{subfigure}
    \begin{subfigure}[t]{0.3\textwidth}
        \centering
        \begin{tikzpicture}[scale=.85]

            \node[circle, fill=Black, scale=0.7] (A) at (-1, -1) {};
            \node[circle, fill=Black, scale=0.7] (B) at (-1.5, 0.25) {};
            \node[circle, fill=Black, scale=0.7] (C) at (-1.5, 2) {};
            \node[circle, fill=Black, scale=0.7] (D) at (1, 4) {};
            \node[circle, fill=Black, scale=0.7] (E) at (2, -.3) {};

            \draw[ultra thick, fill=CornflowerBlue, opacity=0.7] (A.center) -- (B.center) -- (C.center) -- (D.center) -- (E.center) -- cycle;

            \node[circle, fill=Black, scale=0.7] (A) at (-1, -1) {};
            \node[circle, fill=Black, scale=0.7] (B) at (-1.5, 0.25) {};
            \node[circle, fill=Black, scale=0.7] (C) at (-1.5, 2) {};
            \node[circle, fill=Black, scale=0.7] (D) at (1, 4) {};
            \node[circle, fill=Black, scale=0.7] (E) at (2, -.3) {};
        \end{tikzpicture}
        \subcaption*{A 2-dimensional polytope}

        \subcaption*{A 2-dimensional polytope}
    \end{subfigure}
    \begin{subfigure}[t]{0.4\textwidth}
        \centering
        \tdplotsetmaincoords{75}{55}
        \begin{tikzpicture}[tdplot_main_coords,line join=round, scale=.95]

            \path
            (0,0,0)  coordinate (O)
            (0,0,4) coordinate (Z)
            (4,0,0)  coordinate (X)
            (0,4,0)  coordinate (Y);


            \draw[ultra thick,fill=Cyan, opacity=0.5] (O) -- (Y) -- (Z) -- cycle;
            \draw[ultra thick,fill=NavyBlue, opacity=0.5] (O) -- (X)  -- (Y) -- cycle;
            \draw[ultra thick,fill=Gray, opacity=0.5] (O) -- (X)  -- (Z) -- cycle;
            \draw[ultra thick,fill=White, opacity=0.5] (X) -- (Y)  -- (Z) -- cycle;

            \foreach \p in {O,X,Y,Z}
            \draw[fill=black] (\p) circle (2.5pt);

        \end{tikzpicture}
        \subcaption*{A 3-dimensional polytope}
    \end{subfigure}
    \caption{A sampling of polytopes; note that we've only highlighted the vertices, which are the minimal set of points that generate the polytope }
\end{figure}

We will use a similar definition for cones.
They are built out of a finite collection of generating vectors.
\begin{definition}[Cone]\th\label{def:cone}
    Let \( C \subseteq \R^n \).
    We call \( C \) a \emph{cone} if it is the conic combination of finitely many vectors \( x_1, x_2, \dots, x_k \).
    We write this
    \[
        C = \cone(\{x_1, \dots, x_k\}),
    \]
    where
    \[
        \cone(\{x_1, \dots, x_k\}) = \left\{ \lambda_1 x_1 + \lambda_2 x_2 + \cdots + \lambda_k x_k \; \middle| \; \lambda_i \geq 0 \right\}
    \]
    is the conic combinations of \( x_1, x_2, \dots, x_k \).

\end{definition}
Unlike the more familiar notion of cones, these are not pointed cylinders.
\begin{figure}[H]
    \centering
    \begin{subfigure}[t]{0.3\textwidth}
        \centering
        \begin{tikzpicture}[scale=2]

            % Axes 
            \draw[->, Black] (0,0) -- (1,0);
            \draw[->, Black] (0,0) -- (-1,0);
            \draw[->, Black] (0,0) -- (0,1);
            \draw[->, Black] (0,0) -- (0,-1);
            \node[circle, fill=Black, inner sep=0.81] at (0,0) {};

            % Cone
            \draw[fill=CarnationPink, opacity=0.6] (0,0) -- (-0.55, 1.32) -- (-1.26,0.525) -- cycle;
            \draw[->, Black, ultra thick] (0,0) -- (-.5, 1.2);
            \draw[->, Black, ultra thick] (0,0) -- (-1.2,.5);

            \node[circle, fill=Black, inner sep=1.2] at (0,0) {};
        \end{tikzpicture}

        \subcaption*{A 2-dimensional cone}
    \end{subfigure}
    \begin{subfigure}[t]{0.3\textwidth}
        \centering
        \tdplotsetmaincoords{68}{55}
        \begin{tikzpicture}[scale=3,tdplot_main_coords]

            \draw [<->] (-0.5,0,0) -- (1,0,0) node [right] {};
            \draw [<->] (0,-0.5,0) -- (0,1,0) node [left] {};
            \draw [<->] (0,0,-0.5) -- (0,0,1) node [left] {};

            \coordinate (O) at (0,0,0);
            \coordinate (A) at (0.75, 0.75, 0);
            \coordinate (B) at (0.0, 0.75, 0.8);
            \coordinate (C) at (0.7, 0.5, 0.2);

            \node[circle, fill=Black, inner sep=0.81] at (O) {};

            \draw[draw=blue!20,fill=CarnationPink,fill opacity=0.3]  (O)-- (A) -- (B) -- cycle;
            \draw[draw=blue!20,fill=CarnationPink,fill opacity=0.4]  (O)-- (A) -- (C) -- cycle;
            \draw[draw=blue!20,fill=Gray,fill opacity=0.3]  (O)-- (A) -- (C) -- cycle;
            \draw[draw=blue!20,fill=CarnationPink,fill opacity=0.4]  (O)-- (B) -- (C) -- cycle;
            \draw[draw=blue!20,fill=White,fill opacity=0.3]  (O)-- (B) -- (C) -- cycle;

            \draw[ultra thick, ->] (O) -- (A);
            \draw[ultra thick, ->] (O) -- (B);
            \draw[ultra thick, ->] (O) -- (C);

            \node[circle, fill=Black, inner sep=1.2] at (0,0) {};
        \end{tikzpicture}

        \subcaption*{A 3-dimensional cone \\ (generated by 3 vectors)}
    \end{subfigure}
    \begin{subfigure}[t]{0.3\textwidth}
        \centering
        \tdplotsetmaincoords{79}{35}
        \begin{tikzpicture}[scale=3,tdplot_main_coords]

            \draw [<->] (-1,0,0) -- (0.5,0,0) node [right] {};
            \draw [<->] (0,-1,0) -- (0,0.5,0) node [left] {};
            \draw [<->] (0,0,-1) -- (0,0,0.5) node [left] {};

            \coordinate (O) at (0,0,0);
            \coordinate (A) at (-0.75, -0.75, 0);
            \coordinate (B) at (0.0, -0.75, -0.8);
            \coordinate (C) at (-0.7, -0.5, 0.2);
            \coordinate (D) at (0.3, -1.2, -0.3);


            % \node at (A) {A};
            % \node at (B) {B};
            % \node at (C) {C};
            % \node at (D) {D};

            \node[circle, fill=Black, inner sep=0.81] at (O) {};


            \draw[draw=blue!20,fill=CarnationPink,fill opacity=0.3]  (O)-- (A) -- (C) -- cycle;
            \draw[draw=blue!20,fill=White,fill opacity=0.5]  (O)-- (A) -- (C) -- cycle;
            \draw[draw=blue!20,fill=CarnationPink,fill opacity=0.3]  (O)-- (A) -- (B) -- cycle;
            \draw[draw=blue!20,fill=Gray,fill opacity=0.3]  (O)-- (A) -- (B) -- cycle;
            \draw[draw=blue!20,fill=CarnationPink,fill opacity=0.3]  (O)-- (B) -- (D) -- cycle;
            \draw[draw=blue!20,fill=CarnationPink,fill opacity=0.3]  (O)-- (C) -- (D) -- cycle;

            % \draw[draw=blue!20,fill=CarnationPink,fill opacity=0.4]  (O)-- (A) -- (C) -- cycle;
            % \draw[draw=blue!20,fill=Gray,fill opacity=0.3]  (O)-- (A) -- (C) -- cycle;

            % \draw[draw=blue!20,fill=CarnationPink,fill opacity=0.4]  (O)-- (B) -- (C) -- cycle;
            % \draw[draw=blue!20,fill=White,fill opacity=0.3]  (O)-- (B) -- (C) -- cycle;

            \draw[ultra thick, ->,opacity=0.5] (O) -- (D);
            \draw[ultra thick, ->,opacity=0.5, color=CarnationPink] (O) -- (D);
            \draw[ultra thick, ->] (O) -- (A);
            \draw[ultra thick, ->] (O) -- (B);
            \draw[ultra thick, ->] (O) -- (C);

            \node[circle, fill=Black, inner sep=1.2] at (0,0) {};
        \end{tikzpicture}

        \subcaption*{A 3-dimensional cone \\ (generated by 4 vectors)}
    \end{subfigure}
    \caption{Some examples of cones}
    \label{fig:cones}
\end{figure}
We notice that conic combinations are essentially the span of the generating vectors but taking only non-negative linear combinations.
Indeed, one can quickly confirm that \( \cone(\{x_1, \dots, x_k\}) \subseteq \Span(\{x_1, \dots, x_k\}) \).

\subsection{Points vs. Vectors: An Affine Primer}
We have been, and will be going forward, using the words ``points'' and ``vectors'' quite interchangeably.
Is there a difference?
Strictly speaking, yes there is.
Points imply elements of an affine space, while vectors, naturally, are elements of a vector space.
Affine spaces can be thought of as vector spaces where the \( 0 \)-vector is ``forgotten'', but are otherwise essentially the same collection of ``stuff''.
Mathematicians love to make multiple objects out of the same basic thing by giving (or losing) some extra structure.

This poses a slight problem since we want to refer to the dimension of our polytopes and cones (and already have been in figures), but a notion of dimension usually relies on a vector space.
To settle a notion of dimension here, we first want to define what an affine span (also called an affine hull) is.
\begin{definition}[Affine Span]\th\label{def:affineSpan}
    Let \( S \subseteq \R^d \).
    The \emph{affine span} of \( S \) is the set
    \[
        \aff(S) = \left\{ \sum_{i=1}^k \lambda_k x_k \; \middle| \; k > 0,\,  x_i \in S,\, \sum_{i=1}^k \lambda_i = 1 \right\}.
    \]
\end{definition}
The affine span of some set will be something that looks like a linear subspace, but might not include the origin.
In fact if \( 0 \in S \), then the affine span and our more traditional linear span coincide exactly.
The insight here then is that we could always translate our affine span so that it includes the origin.
Once we have a linear subspace we can just use the linear algebra definition of dimension.
\begin{definition}[Affine Dimension]\th\label{def:affineDimension}
    Given any set \( S \subseteq \R^d \), designate some element \( x_0 \in S \).
    The \emph{dimension} of \( S \) is
    \[
        \dim(S) = \dim\bigl( \aff(S)-x_0 \bigr),
    \]
    where, on the right-hand side, \( \dim \) is the standard notion of dimension of a subspace in linear algebra.
\end{definition}
We're overloading our notation a bit, but we promise this mostly reduces cognitive load.
This also goes to explain our switching between ``point'' or ``vector''.
Since our cones are defined to always include \( 0 \), affine and linear terms mostly line up and we are mostly safe to just think in terms of vector spaces.
Indeed, for cones we will always just write \( \Span(\cC) \) in lieu of \( \aff(\cC) \).

Dimension will mostly align with intuition, but it's good to have the definitions at hand if ever in doubt.
Early chapters of \cite{zieglerLecturesPolytopes1995} and \cite{grunbaumConvexPolytopes2003} both provide a good treatment of affine spaces.

\subsection{The Minkowski Sum}
Now that we have the basic shapes down we need be able to make new ones out of existing ones.
The two general strategies here will be to combine them in to new ones and to break them down.
We'll start with learning how we can add shapes together, using what we call the Minkowski sum.
\begin{definition}[Minkowski Sum]\th\label{def:MinkowskiSum}
    Let \( P, Q \subseteq \R^n \).
    The \emph{Minkowski sum} of \( P \) and \( Q \) is given by
    \[
        P + Q = \left\{ p + q \; \middle| \; p \in P, q \in Q \right\}.
    \]
\end{definition}
Sit with this definition for a few moments to confirm the Minkowski sum does have the nice properties of sums we normally expect.
It is commutative, associative, and has an identity in \( \{ 0 \} \).
An additional feature of the definition is that the empty set has the property that for any \( P \subseteq \R^n \)
\[
    P + \emptyset = \emptyset.
\]

A way to think about the Minkowski sum is ``smearing'' one shape around the other.
You pick some point in either shape, then drag it along the boarder of the other shape in the sum.
This makes more sense with a picture.
\begin{figure}[H]
    \centering
    \def\radius{.5cm}
    \begin{tikzpicture}[scale=2]
        \coordinate (a) at (0,0);
        \coordinate (b) at (3,-1);
        \coordinate (c) at (2,2);

        \coordinate (a_out) at ($(a)!\radius!-90:(b)$);
        \coordinate (a_in) at ($(a)!\radius!90:(b)$);
        \coordinate (b_out) at ($(b)!\radius!-90:(b)$);

        % Draw circles! 
        \foreach \nd in {a, b, c} {\draw[dashed, thick, color=Emerald, opacity=1] (\nd) circle (\radius);}

        % Fill side rectangles
        \draw[draw=none, fill=Emerald, opacity=0.3] ($(a)!\radius!-90:(b)$) coordinate (AtB) -- ($(b)!\radius!90:(a)$) coordinate (BtA) -- (b) -- (a) -- cycle;
        \draw[draw=none, fill=Emerald, opacity=0.3] ($(a)!\radius!90:(c)$) coordinate (AtC) -- ($(c)!\radius!-90:(a)$) coordinate (CtA) -- (c) -- (a) -- cycle;
        \draw[draw=none, fill=Emerald, opacity=0.3] ($(b)!\radius!-90:(c)$) coordinate (BtC) -- ($(c)!\radius!90:(b)$) coordinate (CtB) -- (c) -- (b) -- cycle;

        % Fill missing angles and outer part of circle
        \draw pic [draw=none, fill=Emerald, opacity=0.3, angle radius=2*\radius] {angle=AtC--a--AtB};
        \draw pic [draw, ultra thick, angle radius=2*\radius] {angle=AtC--a--AtB};

        \draw pic [draw=none, fill=Emerald, opacity=0.3, angle radius=2*\radius] {angle=BtA--b--BtC};
        \draw pic [draw, ultra thick, angle radius=2*\radius] {angle=BtA--b--BtC};

        \draw pic [draw=none, fill=Emerald, opacity=0.3, angle radius=2*\radius] {angle=CtB--c--CtA};
        \draw pic [draw, ultra thick,  angle radius=2*\radius] {angle=CtB--c--CtA};
        % Draw lines connecting circles
        \draw[ultra thick] ($(a)!\radius!-90:(b)$) -- ($(b)!\radius!90:(a)$) ($(b)!\radius!-90:(c)$) -- ($(c)!\radius!90:(b)$) ($(c)!\radius!-90:(a)$) -- ($(a)!\radius!90:(c)$) ($(a)!\radius!-90:(b)$);

        % Fill inner triangle
        \draw[dashed, thick, fill=RoyalBlue, opacity=0.3] (a) -- (b) -- (c) -- cycle;

        % Put a dot in the center of each circle
        \foreach \nd in {a, b, c} {\node[circle, fill=Black, opacity=1, scale=0.2] at (\nd) {};}
    \end{tikzpicture}
    \caption{A visual representation of a Minkowski sum of a triangle and a circle}
\end{figure}
This is a helpful visual intuition, but it may take a moment to be convinced that, up to translation, the resulting shape doesn't depend on either the shape you choose to smear or the choice of point in that shape.

With the Minkowski sum, we can finally define a general polyhedron.
\begin{definition}[Polyhedron]\th\label{def:polyhedron}
    Let \( P \subseteq \R^n \). We call \( P \) a \emph{polyhedron} if
    \[
        P = \conv(\{x_1, \dots, x_k\}) + \cone(\{y_1, \dots, y_\ell\}),
    \]
    for some finite sets \( \{x_1, \dots, x_k\}, \{y_1, \dots, y_\ell\} \subseteq \R^n \).
\end{definition}
A polyhedron is the result of the Minkowski sum of a polytope and a cone.
Clearly every polytope and cone are polyhedra themselves, as \( \conv(\{0\}) = \cone(\{0\}) = \{ 0 \} \).
Polyhedra are not necessarily bounded, which may seem a bit unusual to those who have seen the word in other contexts.
Moreover, all bounded polyhedra are polytopes, which is not necessarily obvious, but useful to keep in mind.
We will mostly be focused on either polytopes or cones at any one time, but having a more general object that includes both makes our definitions going forward cleaner.

\subsection{About Faces}
Given a polyhedron \( P \), we also get a whole family of polyhedron, the faces of \( P \).
Let's first go back to simpler times.
If we were to think of a cube, we would have faces of the cube as the 2 dimensional squares that make up the sides.
We'd then call the line segments where any two of those squares meet \textit{edges} and the points those edges meet \textit{vertices}.
\begin{figure}[H]
    \centering
    \begin{subfigure}{0.24\textwidth}
        \newcommand{\Depth}{2}
        \newcommand{\Height}{2}
        \newcommand{\Width}{2}
        \begin{tikzpicture}
            \coordinate (O) at (0,0,0);
            \coordinate (A) at (0,\Width,0);
            \coordinate (B) at (0,\Width,\Height);
            \coordinate (C) at (0,0,\Height);
            \coordinate (D) at (\Depth,0,0);
            \coordinate (E) at (\Depth,\Width,0);
            \coordinate (F) at (\Depth,\Width,\Height);
            \coordinate (G) at (\Depth,0,\Height);

            \draw[Red,fill=Red!80] (O) -- (C) -- (G) -- (D) -- cycle;% Bottom Face
            \draw[Red,fill=Red!30] (O) -- (A) -- (E) -- (D) -- cycle;% Back Face
            \draw[Red,fill=Red!10] (O) -- (A) -- (B) -- (C) -- cycle;% Left Face
            \draw[Red,fill=Red!20,opacity=0.8] (D) -- (E) -- (F) -- (G) -- cycle;% Right Face
            \draw[Red,fill=Red!20,opacity=0.6] (C) -- (B) -- (F) -- (G) -- cycle;% Front Face
            \draw[Red,fill=Red!20,opacity=0.8] (A) -- (B) -- (F) -- (E) -- cycle;% Top Face

            %% Following is for debugging purposes so you can see where the points are
            %% These are last so that they show up on top
            %\foreach \xy in {O, A, B, C, D, E, F, G}{
            %    \node at (\xy) {\xy};
            %}
        \end{tikzpicture}
    \end{subfigure}
    \begin{subfigure}{0.24\textwidth}
        \newcommand{\Depth}{2}
        \newcommand{\Height}{2}
        \newcommand{\Width}{2}
        \begin{tikzpicture}
            \coordinate (O) at (0,0,0);
            \coordinate (A) at (0,\Width,0);
            \coordinate (B) at (0,\Width,\Height);
            \coordinate (C) at (0,0,\Height);
            \coordinate (D) at (\Depth,0,0);
            \coordinate (E) at (\Depth,\Width,0);
            \coordinate (F) at (\Depth,\Width,\Height);
            \coordinate (G) at (\Depth,0,\Height);

            \fill[Gray,fill=Gray!50, opacity=0.5] (O) -- (C) -- (G) -- (D) -- cycle;% Bottom Face
            \fill[Gray,fill=Gray!10, opacity=0.5] (O) -- (A) -- (E) -- (D) -- cycle;% Back Face
            \fill[Gray,fill=Gray!5, opacity=0.5] (O) -- (A) -- (B) -- (C) -- cycle;% Left Face
            \fill[Gray,fill=Gray!8,opacity=0.4] (D) -- (E) -- (F) -- (G) -- cycle;% Right Face
            \fill[Gray,fill=Gray!10,opacity=0.2] (C) -- (B) -- (F) -- (G) -- cycle;% Front Face
            \fill[Gray,fill=Gray!10,opacity=0.4] (A) -- (B) -- (F) -- (E) -- cycle;% Top Face

            \draw[Red,fill=Red!70,opacity=0.75] ($ (O) + (0,-0.25*\Width, 0) $) -- ($ (C) + (0,-0.25*\Width, 0) $) -- ($ (G) + (0,-0.25*\Width, 0) $) -- ($ (D) + (0,-0.25*\Width, 0) $) -- cycle;% Bottom Face
            % \draw[Red,fill=Red!30] (O) -- (A) -- (E) -- (D) -- cycle;% Back Face
            \draw[Red,fill=Red!30] ($ (O) + (0.25\Depth,0.25*\Width, 0) $) -- ($ (A) + (0.25\Depth,0.25*\Width, 0) $) -- ($ (E) + (0.25\Depth,0.25*\Width, 0) $) -- ($ (D) + (0.25\Depth,0.25*\Width, 0) $) -- cycle;% Back Face
            % \draw[Red,fill=Red!10] (O) -- (A) -- (B) -- (C) -- cycle;% Left Face
            \draw[Red,fill=Red!30] ($ (O) + (-0.25\Depth,0*\Width, 0) $) -- ($ (A) + (-0.25\Depth,0*\Width, 0) $) -- ($ (B) + (-0.25\Depth,0*\Width, 0) $) -- ($ (C) + (-0.25\Depth,0*\Width, 0) $) -- cycle;% Left Face
            % \draw[Red,fill=Red!20,opacity=0.8] (D) -- (E) -- (F) -- (G) -- cycle;% Right Face
            \draw[Red,fill=Red!30,opacity=0.8] ($ (D) + (0.2\Depth,0*\Width, 0) $) -- ($ (E) + (0.2\Depth,0*\Width, 0) $) -- ($ (F) + (0.2\Depth,0*\Width, 0) $) -- ($ (G) + (0.2\Depth,0*\Width, 0) $) -- cycle;% Right Face
            \draw[Red,fill=Red!20,opacity=0.6] (C) -- (B) -- (F) -- (G) -- cycle;% Front Face
            % \draw[Red,fill=Red!20,opacity=0.8] (A) -- (B) -- (F) -- (E) -- cycle;% Top Face
            \draw[Red,fill=Red!30,opacity=0.8] ($ (A) + (0*\Depth,0.2*\Width, 0*\Height) $) -- ($ (B) + (0*\Depth,0.2*\Width, 0*\Height) $) -- ($ (F) + (0*\Depth,0.2*\Width, 0*\Height) $) -- ($ (E) + (0*\Depth,0.2*\Width, 0*\Height) $) -- cycle;% Right Face

            %% Following is for debugging purposes so you can see where the points are
            %% These are last so that they show up on top
            %\foreach \xy in {O, A, B, C, D, E, F, G}{
            %    \node at (\xy) {\xy};
            %}
        \end{tikzpicture}
    \end{subfigure}
    \begin{subfigure}{0.24\textwidth}
        \newcommand{\Depth}{2}
        \newcommand{\Height}{2}
        \newcommand{\Width}{2}
        \begin{tikzpicture}
            \coordinate (O) at (0,0,0);
            \coordinate (A) at (0,\Width,0);
            \coordinate (B) at (0,\Width,\Height);
            \coordinate (C) at (0,0,\Height);
            \coordinate (D) at (\Depth,0,0);
            \coordinate (E) at (\Depth,\Width,0);
            \coordinate (F) at (\Depth,\Width,\Height);
            \coordinate (G) at (\Depth,0,\Height);

            \coordinate(Down) at (0, -0.1*\Width, 0);
            \coordinate(Up) at (0, 0.1*\Width, 0);
            \coordinate(Right) at (0.1*\Depth, 0, 0);
            \coordinate(Left) at (-0.1*\Depth, 0, 0);
            \coordinate(Forward) at (0, 0, 0.1*\Height);
            \coordinate(Back) at (0, 0, -0.1*\Height);

            \fill[ Gray,fill=Gray!80, opacity=0.5] (O) -- (C) -- (G) -- (D) -- cycle;% Bottom Face
            \fill[Gray,fill=Gray!50, opacity=0.5] (O) -- (A) -- (E) -- (D) -- cycle;% Back Face
            \fill[Gray,fill=Gray!10, opacity=0.5] (O) -- (A) -- (B) -- (C) -- cycle;% Left Face
            \fill[Gray,fill=Gray!20,opacity=0.4] (D) -- (E) -- (F) -- (G) -- cycle;% Right Face
            \fill[Gray,fill=Gray!20,opacity=0.2] (C) -- (B) -- (F) -- (G) -- cycle;% Front Face
            \fill[Gray,fill=Gray!20,opacity=0.4] (A) -- (B) -- (F) -- (E) -- cycle;% Top Face

            % Back Face
            \draw[Red, ultra thick, opacity=0.5] ($ (O) + (Back) + (Left) $) -- ($ (A) + (Back) + (Left) $);
            \draw[Red, ultra thick, opacity=0.5] ($ (A) + (Back) + (Up) $) -- ($ (E) + (Back) + (Up) $);
            \draw[Red, ultra thick, opacity=0.5] ($ (E) + (Back) + (Right) $) -- ($ (D) + (Back) + (Right) $);
            \draw[Red, ultra thick, opacity=0.5] ($ (D) + (Back) + (Down) $) -- ($ (O) + (Back) + (Down) $);
            % Left Side
            \draw[Red, ultra thick, opacity=0.5] ($ (A) + (Up) + (Left) $) -- ($ (B) + (Up) + (Left) $);
            \draw[Red, ultra thick, opacity=0.5] ($ (O) + (Left) + (Down) $) -- ($ (C) + (Left) + (Down) $);
            % Right Side
            \draw[Red, ultra thick, opacity=0.5] ($ (E)  + (Up) + (Right) $) -- ($ (F)  + (Up) + (Right) $);
            \draw[Red, ultra thick, opacity=0.5] ($ (G) + (Down) + (Right) $) -- ($ (D) + (Down) + (Right) $);
            % Front Face
            \draw[Red, ultra thick, opacity=0.5] ($ (C) + (Forward) + (Left) $) -- ($ (B) + (Forward) + (Left) $);
            \draw[Red, ultra thick, opacity=0.5] ($ (B) + (Forward) + (Up) $) -- ($ (F) + (Forward) + (Up) $);
            \draw[Red, ultra thick, opacity=0.5] ($ (F) + (Forward) + (Right) $) -- ($ (G) + (Forward) + (Right) $);
            \draw[Red, ultra thick, opacity=0.5] ($ (G) + (Forward) + (Down) $) -- ($ (C) + (Forward) + (Down) $);

            % \draw[Red, thick] ($ (B) + -0.1*(B) $) -- ($ (C) + -0.1*(C) $);
            % \draw[Red, thick] ($ (B) + -0.1*(B) $) -- ($ (C) + -0.1*(C) $);
            % \draw[Red, thick] ($ (B) + -0.1*(B) $) -- ($ (C) + -0.1*(C) $);
            % \draw[Red, thick] ($ (B) + -0.1*(B) $) -- ($ (C) + -0.1*(C) $);

            %% Following is for debugging purposes so you can see where the points are
            %% These are last so that they show up on top
            %\foreach \xy in {O, A, B, C, D, E, F, G}{
            %    \node at (\xy) {\xy};
            %}
        \end{tikzpicture}
    \end{subfigure}
    \begin{subfigure}{0.24\textwidth}
        \newcommand{\Depth}{2}
        \newcommand{\Height}{2}
        \newcommand{\Width}{2}
        \begin{tikzpicture}
            \coordinate (O) at (0,0,0);
            \coordinate (A) at (0,\Width,0);
            \coordinate (B) at (0,\Width,\Height);
            \coordinate (C) at (0,0,\Height);
            \coordinate (D) at (\Depth,0,0);
            \coordinate (E) at (\Depth,\Width,0);
            \coordinate (F) at (\Depth,\Width,\Height);
            \coordinate (G) at (\Depth,0,\Height);

            \draw[Gray,fill=Gray!50, opacity=0.5] (O) -- (C) -- (G) -- (D) -- cycle;% Bottom Face
            \draw[Gray,fill=Gray!10, opacity=0.5] (O) -- (A) -- (E) -- (D) -- cycle;% Back Face
            \draw[Gray,fill=Gray!5, opacity=0.5] (O) -- (A) -- (B) -- (C) -- cycle;% Left Face
            \draw[Gray,fill=Gray!8,opacity=0.4] (D) -- (E) -- (F) -- (G) -- cycle;% Right Face
            \draw[Gray,fill=Gray!10,opacity=0.2] (C) -- (B) -- (F) -- (G) -- cycle;% Front Face
            \draw[Gray,fill=Gray!10,opacity=0.4] (A) -- (B) -- (F) -- (E) -- cycle;% Top Face

            \node[circle, fill=Red, opacity=0.5, minimum size=5pt, inner sep=0] at (O) {};
            \node[circle, fill=Red, opacity=0.5, minimum size=5pt, inner sep=0] at (A) {};
            \node[circle, fill=Red, opacity=0.5, minimum size=5pt, inner sep=0] at (B) {};
            \node[circle, fill=Red, opacity=0.5, minimum size=5pt, inner sep=0] at (C) {};
            \node[circle, fill=Red, opacity=0.5, minimum size=5pt, inner sep=0] at (D) {};
            \node[circle, fill=Red, opacity=0.5, minimum size=5pt, inner sep=0] at (E) {};
            \node[circle, fill=Red, opacity=0.5, minimum size=5pt, inner sep=0] at (F) {};
            \node[circle, fill=Red, opacity=0.5, minimum size=5pt, inner sep=0] at (G) {};

            % \draw[Red,fill=Red!80] (O) -- (C) -- (G) -- (D) -- cycle;% Bottom Face
            % \draw[Red,fill=Red!30] (O) -- (A) -- (E) -- (D) -- cycle;% Back Face
            % \draw[Red,fill=Red!10] (O) -- (A) -- (B) -- (C) -- cycle;% Left Face
            % \draw[Red,fill=Red!20,opacity=0.8] (D) -- (E) -- (F) -- (G) -- cycle;% Right Face
            % \draw[Red,fill=Red!20,opacity=0.6] (C) -- (B) -- (F) -- (G) -- cycle;% Front Face
            % \draw[Red,fill=Red!20,opacity=0.8] (A) -- (B) -- (F) -- (E) -- cycle;% Top Face

            %% Following is for debugging purposes so you can see where the points are
            %% These are last so that they show up on top
            %\foreach \xy in {O, A, B, C, D, E, F, G}{
            %    \node at (\xy) {\xy};
            %}
        \end{tikzpicture}
    \end{subfigure}
    \caption{The faces of a cube; note that the full cube is a face of itself (as is the empty set, which we drew as accurately as possible)}
\end{figure}
If this sounds familiar then the intuition for our more general notion of the face of a polyhedron is not far.
Back in our world of polyhedral geometry, we know that a cube is a polyhedron and that each of those squares, lines, and points are also themselves polyhedron.
Informally, we use the term face to describe all these ``sub-polyhedra'' that make up the boundary of a polyhedron.
\begin{definition}[Face]\th\label{def:face}
    Let \( P \subseteq \R^n \) be a polyhedron and fix an inner product \( \ast \in \inn(\R^n) \).
    Recall that the hyperplane normal to a vector \( x \in \R^n \) at distance \( q \in \R \)  is given by
    \[
        \HP_x(b) = \left\{ v \in \R^n \; \middle| \; v \ast x = b \right\}.
    \]
    Additionally, any hyperplane defines a lower half-spaces given by
    \[
        \LH_x(b) = \left\{ v \in \R^n \; \middle| \; v \ast x \leq b \right\},
    \]
    respectively.

    Then \( F \subseteq P \) is a \emph{face} of \( P \) if there exist \( x, b \) such that
    \[
        F = P \cap \HP_x(b)
    \]
    such that \( P \) lies entirely in the lower half-space of \( \HP_x(b) \); i.e.,
    \begin{gather*}
        P \subseteq \LH_x(b).
    \end{gather*}

\end{definition}
Some notable quirks of this definition are that given any \( x \in \R^n \), \( F = \HP_x(0) \cap P = P \) and so \( P \) is face of itself.
Likewise, there exist infinitely many hyperplanes such that \( \HP_x(b) \cap P = \emptyset \), which means the empty set too is a face of any polyhedron \( P \).
As a note, we do still call the \( 0 \)-dimensional faces vertices, while the generic term for a \( (d - 1) \)-dimensional face of a \( d \)-dimensional polyhedron is a \textit{facet}.
\begin{figure}[H]
    \centering
    \tdplotsetmaincoords{70}{110} % Adjust the viewing angle
    \begin{tikzpicture}[tdplot_main_coords, scale=1.5]

        % Cube
        \draw[thick] (0,0,0) -- (2,0,0) -- (2,2,0) -- (0,2,0) -- cycle;
        \draw[thick] (0,0,2) -- (2,0,2) -- (2,2,2) -- (0,2,2) -- cycle;
        \draw[thick] (0,0,0) -- (0,0,2);
        \draw[thick] (2,0,0) -- (2,0,2);
        \draw[thick] (2,2,0) -- (2,2,2);
        \draw[thick] (0,2,0) -- (0,2,2);

        % Plane
        \fill[gray!50, opacity=0.5] (-0.75,-0.75,0) -- (2.75,-0.75,0) -- (2.75,2.75,0) -- (-0.75,2.75,0) -- cycle;

        % Intersection edge
        \draw[ultra thick, Red, fill=Red, opacity=0.5] (0,0,0) -- (2,0,0) -- (2,2,0) -- (0,2,0) -- cycle;

    \end{tikzpicture}
    \caption{Example of specifying a facet of our cube with a plane}
\end{figure}

From our cube intuition exercise, there are some things about faces that we would hope to generalize to our broader notion of faces.
We'll state these without proof, and again refer to the early chapters of \cite{zieglerLecturesPolytopes1995}.
\begin{proposition}
    Given a polyhedron \( P \subseteq R^n \), every face \( F \subseteq P \) is a polyhedron.
    Better, if \( P \) is a polytope then \( F \) is a polytope, and if \( P \) is a cone then \( F \) is a cone.
\end{proposition}
\begin{proposition}
    Let \( P \subseteq \R^n \) be a polyhedron, and \( F \subseteq P \) a face of \( P \).
    From above, we know that \( F \) is a polyhedron.
    Then for any \( F' \subseteq F \) a face of \( F \), we have \( F' \) is also a face of \( P \).
\end{proposition}
\begin{proposition}
    Let \( P \subseteq \R^n \) be a polyhedron and \( F_1, F_2 \subseteq P \) be two faces of \( P \).
    Then \( F_1 \cap F_2 \) is a face of \( P \).
\end{proposition}
It's worth remembering that the intersection of two faces can be empty, but \( \emptyset \subseteq P \) is a face of any polyhedron \( P \), so this causes no issues.

We will often write \( F \preceq P \) to mean \( F \) is a face of \( P \).
Indeed, the relation ``is a face of'' induces a partial ordering on the set of all faces of \( P \).
Even stronger, using the propositions above it follows that the relation induces a lattice.

\subsection{Polyhedral Complex}
The last geometric structure we need as background consist of particular collections of polyhedra, known as polyhedral complexes.
\begin{definition}[Polyhedral Complex]\th\label{def:polyhedralComplex}
    A \emph{polyhedral complex} \( \cC \) is a finite collection of polyhedra in \( \R^n \) such that
    \begin{enumerate}
        \item the empty set is in \( \cC \),
        \item for any polyhedron \( P \in \cC \), all faces of \( P \) are also in \( \cC \),
        \item for any two polyhedra \( P, Q \in \cC \), the intersection \( P \cap Q \in \cC \).
    \end{enumerate}
\end{definition}
We can think of polyhedral complexes as sets of polyhedra that intersect nicely.
We won't be too concerned about complexes of general polyhedra, and instead focus on when our complexes are restricted to either all polytopes or all cones.
\begin{definition}[Polytopal Complex]\th\label{def:polytopalComplex}
    A \emph{polytopal complex} \( \cC \) is a polyhedral complex where every element \( P \in \cC \) is bounded, i.e., a polytope.
\end{definition}
One might guess that next we'll define a conic complex or some such thing.
In fact, a polyhedral complex of only cones is called a \emph{fan}, we assume mostly to torment anyone trying to do a web-search for information on them.
\begin{definition}[Fan]\th\label{def:fan}
    A \emph{fan} \( \bSigma \) is a polyhedral complex where every element \( \sigma \in \bSigma \) is a cone.
\end{definition}
\begin{figure}[H]
    \centering
    \begin{subfigure}{0.48\textwidth}
        \centering
        \begin{tikzpicture}[scale=2.8]
            \draw[draw=blue!10,fill=blue!10, fill opacity=.5]    (.17,.07) -- (.9,.07) -- (.9,.8);
            % \draw[draw=blue!10,fill=blue!10, fill opacity=.5]    (.07,-.07) -- (.9,-.07) -- (.9,-.9) -- (.07,-.9);
            \draw[draw=blue!10,fill=blue!10, fill opacity=.5]    (-.07,-.17) -- (-.07,-.9) -- (-.8,-.9);
            \draw[draw=blue!10,fill=blue!10, fill opacity=.5 ]    (-.17,-.07) -- (-.9,-.07) -- (-.9,-.8);
            % \draw[draw=blue!10,fill=blue!10, fill opacity=.5]    (-.07,.07) -- (-.9,.07) -- (-.9,.9) -- (-.07,.9);
            % \draw[draw=blue!10,fill=blue!10, fill opacity=.5]    (.07,.2) -- (.07,.9) -- (.8,.9);
            \draw[->] (0,0) -- (1,0);
            \draw[->] (0,0) -- (1,1);
            % \draw[->] (0,0) -- (0,1);
            \draw[->] (0,0) -- (-1,0);
            \draw[->] (0,0) -- (-1,-1);
            \draw[->] (0,0) -- (0,-1);
            % \node[right] at (1,0) {$\rho_1$};
            % \node[right] at (1,1) {$\rho_{12}$};
            % \node[right] at (0,1) {$\rho_2$};
            % \node[left] at (-1,0) {$\rho_{02}$};
            % \node[right] at (-1,-1) {$\rho_0$};
            % \node[right] at (0,-1) {$\rho_{01}$};
        \end{tikzpicture}
    \end{subfigure}
    \begin{subfigure}{0.48\textwidth}
        \centering
        \tdplotsetmaincoords{68}{55}
        \begin{tikzpicture}[scale=2.5,tdplot_main_coords]
            \draw[draw=blue!20,fill=blue!20,fill opacity=0.5]  (0,0, 0)-- (1, 0, 0) -- (1, 1, 1) -- cycle;
            \draw[draw=blue!20,fill=blue!20,fill opacity=0.5]  (0,0, 0)-- (0, 1, 0) -- (1, 1, 1) -- cycle;
            \draw[draw=blue!20,fill=blue!20,fill opacity=0.5]  (0,0, 0)-- (0, 0, 1) -- (1, 1, 1) -- cycle;
            \draw[draw=blue!20,fill=blue!20,fill opacity=0.5] (0,0, 0)-- (1, 0, 0) -- (0, -1, -1) -- cycle;
            \draw[draw=blue!20,fill=blue!20,fill opacity=0.5] (0,0, 0)-- (0, 1, 0) -- (-1, 0, -1) -- cycle;
            \draw[draw=blue!20,fill=blue!20,fill opacity=0.5] (0,0, 0)-- (0, 0, 1) -- (-1, -1, 0) -- cycle;
            \draw[draw=blue!20,fill=blue!20,fill opacity=0.5] (0,0, 0)-- (-1, -1, 0) -- (-1, -1, -1) -- cycle;
            \draw[draw=blue!20,fill=blue!20,fill opacity=0.5] (0,0, 0)-- (-1, 0, -1) -- (-1, -1, -1) -- cycle;
            \draw[draw=blue!20,fill=blue!20,fill opacity=0.5] (0,0, 0)-- (0, -1, -1) -- (-1, -1, -1) -- cycle;
            \draw[->] (0,0,0) -- (1,0,0);
            % \node[right] at (1,0,0) {$\rho_1$};
            \draw[->,gray] (0,0,0) -- (0,1,0);
            % \node[above,gray] at (0,0.97,0) {$\rho_2$};
            \draw[->] (0,0,0) -- (0,0,1);
            % \node[above] at (0,0,1) {$\rho_3$};
            \draw[->] (0,0,0) -- (1,1,1);
            % \node[right] at (1,1,1) {$\rho_{123}$};
            \draw[->] (0,0,0) -- (-1,-1,-1);
            % \node[left] at (-1,-1,-1) {$\rho_0$};
            \draw[->] (0,0,0) -- (-1,-1,0);
            % \node[left] at (-1,-1,0) {$\rho_{03}$};
            \draw[->,gray] (0,0,0) -- (-1,0,-1);
            % \node[left,gray] at (-1,0,-1) {$\rho_{02}$};
            \draw[->] (0,0,0) -- (0,-1,-1);
            % \node[below] at (0,-1,-1) {$\rho_{01}$};
            %\draw[dotted] (1,1,1) -- (1,1,-1) -- (1,-1,-1) -- (1,-1,1) -- cycle; 
            %\draw[dotted] (-1,1,1) -- (-1,1,-1) -- (-1,-1,-1) -- (-1,-1,1) -- cycle; 
            %\draw[dotted] (1,1,1) -- (-1,1,1) -- (-1,-1,1) -- (1,-1,1) -- cycle; 
            %\draw[dotted] (1,1,-1) -- (-1,1,-1) -- (-1,-1,-1) -- (1,-1,-1) -- cycle; 
        \end{tikzpicture}
    \end{subfigure}
    \caption{Examples of fans; even though they live in different spaces they are both composed only of fans of dimension 2 and less}
    \label{fig:fans}
\end{figure}
From the title of this chapter, it will come as no surprise that fans are quite important.
As such there's a few properties of fans that we want to introduce.

\subsection{We're Big Fans (of These Properties of Fans)}

We would like to introduce three properties of fans.
These properties tend to make fans nice to work with as we will see.
The first tells us something about the dimension of all the maximal cones in the fan.
\begin{definition}[Pure]\th\label{def:pure}
    A cone \(\sigma \in \bSigma \) is maximal if it is not the proper face of another cone in \( \bSigma \).
    A fan is \emph{pure} if every maximal cone in \( \bSigma \) is the same dimension.

    We say \( \bSigma \) is a \( d \)-fan when it is pure of dimension \( d \).
\end{definition}
Basically, a fan is pure if the largest cones in it are all of the same dimension.
Both fans in figure~\ref{fig:fans} are pure of dimension 2.
We can consider a counterexample as well.
\begin{figure}[H]
    \centering
    \tdplotsetmaincoords{95}{15}
    \begin{tikzpicture}[scale=2.5, tdplot_main_coords, scale=0.8]
        \draw[draw=Red!20,fill=Red,fill opacity=0.5]  (0,0, 0)-- (1, 0, 0) -- (1, 1, 1) -- cycle;
        \draw[draw=Red!20,fill=Red,fill opacity=0.5]  (0,0, 0)-- (0, 0, 1) -- (1, 1, 1) -- cycle;

        \draw[draw=Blue!20,fill=Blue,fill opacity=0.5] (0,0, 0)-- (1, 0, 0) -- (0, -1, -1) -- cycle;
        \draw[draw=White!20,fill=White,fill opacity=0.5] (0,0, 0)-- (1, 0, 0) -- (0, -1, -1) -- cycle;

        \draw[draw=Green!20,fill=Green,fill opacity=0.5] (0,0, 0)-- (0, -1, -1) -- (-1, -1, -1) -- cycle;
        \draw[draw=White!20,fill=White,fill opacity=0.7] (0,0, 0)-- (0, -1, -1) -- (-1, -1, -1) -- cycle;

        \draw[->, thick, black!75] (0,0,0) -- (1,1,1);
        \draw[->, thick] (0,0,0) -- (-1,-1,-1);
        \draw[->, thick] (0,0,0) -- (0,-1,-1);

        \draw[draw=Red!20,fill=Red,fill opacity=0.6]  (0,0, 0)-- (0, 0, 1) -- (1, 0, 0) -- cycle;
        \draw[draw=Red!60,fill=Gray,fill opacity=0.4]  (0,0, 0)-- (0, 0, 1) -- (1, 0, 0) -- cycle;

        \draw[->, thick] (0,0,0) -- (1,0,0);
        \draw[->, thick] (0,0,0) -- (0,0,1);
    \end{tikzpicture}
    \caption{This fan is not pure, since it has maximal cones of both dimension 2 and 3}
\end{figure}
The next property relates to how many rays are necessary to generate each cone.
\begin{definition}[Simplicial Fan]\th\label{def:simplicial}
    A fan \( \Sigma \) is \emph{simplicial} if every cone \( \sigma \in \Sigma \) is simplicial.
    A cone \( \sigma \in \Sigma \) is simplicial if
    \[
        \dim (\sigma)  = |\sigma(1)|,
    \]
    where \( \sigma(1) \) is the set of \( 1 \)-dimensional faces of \( \sigma \).
\end{definition}
Alternatively, a cone is simplicial if its rays form a basis for the subspace that cone spans.
In figure~\ref{fig:cones}, the first two cones are simplicial while the third is not.

Next, we have something that is not specifically a property of fan but rather some convenient extra information.
\begin{definition}[Marked Fan]\th\label{marked}
    Let \( \bSigma \) be a fan.
    We say \( \bSigma \) is a \emph{marked fan} if there is a chosen set
    \[
        \left\{ u_\rho \in \rho \; \middle| \; \rho \in \bSigma(1) \right\}
    \]
    such that \( \cone(u_\rho) = \rho \).
\end{definition}
We may also say that \( \bSigma \) is marked with the points \(  \left\{ u_\rho \in \rho \; \middle| \; \rho \in \bSigma(1) \right\} \).
Since there will always exist some choice of points that mark a fan we may just say the fan is marked, and not worry about specific values.
Otherwise, we'll provide the marking explicitly if necessary.

Given a simplicial \( d \)-fan, it can have yet another property, that of being tropical.
Tropical fans come from the realm of tropical geometry, another deep topic we are going to avoid delving into.
That tropical fans are in play at all relates back to lurking algebraic varieties that we continue to not fully address.
We can, however, at least somewhat easily characterize tropical fans.
They are fans that satisfy the weighted balancing condition.
\begin{definition}[Tropical Fan]\th\label{def:tropical}
    Let \( \bSigma \) be a marked, simplicial \( d \)-fan.
    Given a weight function
    \[
        \omega\,:\; \bSigma(d) \to \R_{>0},
    \]
    we say the pair \( (\bSigma, \omega) \) is a \emph{tropical fan} if for every \( \tau \in \bSigma(d-1) \)
    \[
        \sum_{\mathclap{\substack{\sigma \in \bSigma(d) \\ \tau \preceq \sigma}}} \omega(\sigma) u_{\sigma \setminus \tau} \in \Span(\tau),
    \]
    where \( \sigma \setminus \tau \) is shorthand for the single element in \( \sigma(1) \setminus \tau(1) \).
\end{definition}
We will go on to immediately abuse notation and say a fan \( \bSigma \) is itself tropical, as long there exists a weight function \( \omega \) that satisfies the condition.
The exact implications of this definition takes a moment to parse, but the main takeaway is that it imposes some relationship between the cones of the fan.
If the weight function is particularly simple then we have another term.
\begin{definition}[Balanced Fan]\th\label{def:balanced}
    A tropical fan \( (\bSigma, \omega) \) is \emph{balanced} if the weight function \( \omega \) is the constant function
    \[
        \omega(\sigma) = 1
    \]
    for all \( \sigma \in \bSigma(d) \).
\end{definition}
When a fan is balanced we can omit references to the weight function \( \omega \), a very convenient aspect of having a balanced fan.
We've now built up all the background on shapes we'll need.


\section{The Geometer's Guide to Generating Log-Concave Sequences}
Having developed our shapes, we are going to need to measure them somehow.
Specifically, we want their volume.
As with the rest of this chapter so far, we're going to take something that most people can intuitively grasp for 3-dimensional shapes and generalize the hell out of it.
Not only do we need volume for arbitrary dimensional polyhedra, we need something called the mixed volume which gives us some sense of volume of multiple shapes.
This work however will allow us to introduce a classic result of convex geometry that relates geometry to log-concave sequences.

From here, the last few background points can no longer be readily found in Ziegler, who as a topologist, we suspect without proof, cares little for things like volume.
Instead, we swap out our Germans and recommend Rolf Schneider's ``Convex Bodies'' \cite{schneiderConvexBodiesBrunn2013} as a comprehensive reference.

\subsection{Volume Functions}
We again need to formalize something that most of us would take for granted.
The notion of volume is intuitive enough for 3-dimensional shapes, we however need to generalize this to all dimensions.
We will actually only need the volume of polytopes, so we restrict our notion of volume to just them.
\begin{definition}[Volume Function]
    A \emph{volume function} is a map
    \[
        \Vol_n\,:\; \{\text{polytopes in \( \R^n \)}\} \to \R_{\geq 0}
    \]
    such that for any polytopes \( P,Q \subseteq \R^n \), \( \Vol_n \)
    \begin{enumerate}
        \item{is non-negative:} \( \Vol_n(P) > 0 \) when \( \dim(P) = n \) and \( \Vol_n(P) = 0 \) when \( \dim(P) < n \),
        \item{is translation invariant:} \( \Vol_n(P) =  \Vol_n(P+v) \) for any \( v \in \R^n \),
        \item{respects inclusion-exclusion:} when \( P \cup Q \) is a polytope,
        \[
            \Vol_n(P \cup Q) = \Vol(P) + \Vol_n(Q) - \Vol(P \cap Q),
        \]
        \item{respects linear maps:} for any \( T \in \Hom(\R^n, \R^n) \),
        \[
            \Vol_n(T(P)) = |\det(T)|\Vol_n(P).
        \]
    \end{enumerate}
\end{definition}
When the dimension is unambiguous, we will write \( \Vol_n \) as simply \( \Vol \).
This definition does not uniquely specify a single ``volume function'', but rather a family of maps that all differ from one another by a constant multiple.
We can differentiate different volume maps by which polytope in \( \R^n \) they take to \( 1 \).
For example, in \( 3 \)-dimensions, our ``standard'' volume function is the one that takes the unit cube to \( 1 \).

In general, since any two volume functions only differ by constant, we can choose the volume function that makes our equations easiest to read.
For example, we will be using \textit{simplicial volume} quite a bit.
This is the volume defined such that
\[
    \Vol_n \big(\conv(\{0, e_1, e_2, \dots, e_n\})\big) = 1
\]
where \( e_1, \dots, e_n \) are the basis vectors of our \( n \)-dimensional vector space.

\subsection{Mixed Volume}
While the volume function may not be too odd a concept we will use it to define a less widely known function.
Given any \( 2 \) polytopes in \( \R^2 \), or \( 3 \) polytopes in \( \R^3 \), or more generally \( n \) polytopes in \( \R^n \), we want a map from these collections of polytopes to \( \R_{\geq 0} \) that is, in some sense, consistent with volume.
We call this map the mixed volume function.
\begin{definition}[Mixed Volume -- Characterization]\th\label{def:mixedVolume}
    The \emph{mixed volume function} is a map \( \MVol_n \) from an ordered multiset \( P_1, P_2, \dots, P_n \subseteq \R^n \) of polytopes to \( \mathbb{R}_{\geq 0} \), such that it has the following properties:
    \begin{enumerate}
        \item \( \MVol_n(P, P, \dots, P) = \Vol_n(P) \), for any polytope \( P \subseteq \R^n \),
        \item \( \MVol_n \) is symmetric in all arguments, and
        \item \( \MVol_n \) is multilinear with respect to scaling and Minkowski addition.
    \end{enumerate}
\end{definition}
Just like volume, we'll often just notate this as \( \MVol \) when safe to do so.
The proof that such a function exists and is indeed uniquely defined by these properties can be found in \cite{schneiderConvexBodiesBrunn2013}.
While this characterization is useful, it goes very little of the way to actually telling us what the mixed volume is.

Consider two polytopes \( P, Q \subseteq \R^2 \).
We could ask ourselves, what is the volume of the Minkowski sum of \( P \) and \( Q \).
We could be more ambitious and even allow ourselves to scale \( P \) and \( Q \) by arbitrary values.
That is to say, let us consider the volume of
\[
    \Vol_2(\lambda P + \mu Q),
\]
for some \( \lambda, \mu \in \R\).
Answering this question is where mixed volumes appear as something more concrete.
While not immediately obvious, this volume can always be expressed as a polynomial in \( \lambda \) and \( \mu \), and mixed volumes appear as coefficients of these polynomials.
We can consider an example.
\begin{figure}[H]
    \centering
    \begin{subfigure}[t]{0.3\textwidth}
        \centering
        \begin{tikzpicture} [scale=0.8,
                axis/.style={-, black, very thin},
                vector/.style={-stealth, black, thin},
                flat vector/.style={-stealth, red, very thick},
                vector guide/.style={dashed, red, very thick}]

            \coordinate (0) at (-2, 1) {};
            \coordinate (1) at (-1, 1) {};
            \coordinate (2) at (-2, 0) {};
            \coordinate (3) at (-1, 0) {};
            \coordinate (4) at (0, 2) {};
            \coordinate (5) at (0, 0) {};
            \coordinate (6) at (2, 0) {};
            \coordinate[label=below:\( P \)] (7) at (-1.5, -0.25) {};
            \coordinate[label=below:\( Q \)] (8) at (1, -0.25) {};

            \draw [fill=Cyan, fill opacity=0.5] (0.center) -- (1) -- (3) -- (2) -- cycle;
            \draw [fill=Red, fill opacity=0.5] (4) -- (5) -- (6) -- cycle;
        \end{tikzpicture}
    \end{subfigure}
    \hfill
    \begin{subfigure}[t]{0.3\textwidth}
        \centering
        \begin{tikzpicture} [scale=0.8,
                axis/.style={-, black, very thin},
                vector/.style={-stealth, black, thin},
                flat vector/.style={-stealth, red, very thick},
                vector guide/.style={dashed, red, very thick}]

            \coordinate (0) at (-3, 1) {};
            \coordinate (1) at (-2, 1) {};
            \coordinate (2) at (-3, 0) {};
            \coordinate (3) at (-2, 0) {};

            \coordinate (A) at (-3, 2) {};
            \coordinate (B) at (-1, 2) {};
            \coordinate (C) at (-1, 0) {};

            \coordinate (4) at (0, 2) {};
            \coordinate (5) at (0, 0) {};
            \coordinate (6) at (2, 0) {};

            \coordinate (D) at (0, 3) {};
            \coordinate (E) at (3, 0) {};

            \coordinate[label=below:\( \lambda P \)] (7) at (-2.1, -0.25) {};
            \coordinate[label=below:\( \mu Q \)] (8) at (1.4, -0.25) {};

            \draw [fill=Cyan, fill opacity=0.5] (0.center) -- (1) -- (3) -- (2) -- cycle;
            \draw [fill=Cyan, fill opacity=0.2] (0) -- (A) -- (B) -- (C) -- (3) -- (1) -- cycle;

            \draw [fill=Red, fill opacity=0.5] (4) -- (5) -- (6) -- cycle;
            \draw [fill=Red, fill opacity=0.2] (4) -- (D) -- (E) -- (6) --cycle;
        \end{tikzpicture}
    \end{subfigure}
    \hfill
    \begin{subfigure}[t]{0.35\textwidth}
        \centering
        \begin{tikzpicture} [scale=1.0,
                axis/.style={-, black, very thin},
                vector/.style={-stealth, black, thin},
                flat vector/.style={-stealth, red, very thick},
                vector guide/.style={dashed, red, very thick}]

            \coordinate (0) at (-1, -1) {};
            \coordinate (1) at (-1, 0.5) {};
            \coordinate (2) at (0.5, 0.5) {};
            \coordinate (3) at (0.5, -1) {};

            \coordinate (4) at (-1, 3) {};
            \coordinate (5) at (0.5, 3) {};
            \coordinate (6) at (3, 0.5) {};
            \coordinate (7) at (3, -1) {};

            \coordinate [label=below:\( \lambda P + \mu Q \)] (8) at (1, -1.4) {};

            \draw [fill=Cyan, fill opacity=0.6] (0) -- (1) -- (2) -- (3) -- cycle;
            \draw [fill=Red, fill opacity=0.6] (2) -- (5) -- (6) -- cycle;

            \draw [fill=Lavender, fill opacity=0.4] (1) -- (2) -- (5) -- (4) -- cycle;
            \draw [fill=Lavender, fill opacity=0.4] (2) -- (6) -- (7) -- (3) -- cycle;

        \end{tikzpicture}
    \end{subfigure}
    \caption*{ \( \Vol(\lambda P + \mu Q) = \MVol(P, P)\lambda^2 + 2\MVol(P, Q)\lambda \mu + \MVol(Q,Q) \mu^2 \) }

    \caption{The mixed volume appears in the coefficients of the volume polynomial;\\
        recall that \( \MVol(P, P) = \Vol(P) \)}
\end{figure}
Our example in the figure is nice because it is already symmetrical.
In general this will not be the case and the coefficients will need to be rewritten to be symmetric, though this can always be done.
This idea generalizes to any dimension, and can be taken as another definition of mixed volume.
\begin{definition}[Mixed Volume -- As Coefficients]
    Let  \( P_1, P_2, \dots, P_\ell \subseteq \R^n \) be polytopes.
    The function
    \[
        f(\lambda_1, \lambda_2, \dots, \lambda_\ell)  = \Vol(\lambda_1 P_1 + \lambda_2 P_2 + \cdots + \lambda_\ell P_\ell), \quad \lambda_j \geq 0
    \]
    is a homogeneous polynomial of degree \( n \).
    It can be written symmetrically as
    \[
        f(\lambda_1, \dots, \lambda_\ell) = \sum_{\mathclap{j_1, j_2, \dots, j_n = 1}}^{\ell} \MVol(P_{j_1}, \dots, P_{j_n})\lambda_{j_1} \cdots \lambda_{j_n}.
    \]

    The coefficient associated to \( \lambda_{j_1} \cdots \lambda_{j_n} \) is the \emph{mixed volume} of \( P_{j_1}, \dots, P_{j_n} \).

\end{definition}
Of course, you could start with either definition of mixed volume and derive the other, they are equivalent after all.

At this point, one may begin to wonder why this section should even exist.
We seem to have gone rather far afield with our geometry lesson.
Remember that our ultimate goal is to show something is a log-concave sequence.
Mixed volumes are the key to a method generating log-concave sequences via geometry.

\subsection{The Alexandrov–Fenchel Inequality}
Finally, we conclude with an important classic result in convex geometry.
Proved by Alexandr Alexandrov in \cite{aleksandrovZurTheorieGemischter1937}, with a contemporaneous but not quite accurate proof by Werner Fenchel, this theorem gives us a fundamental relationship of mixed volumes of convex bodies.
We again restrict ourselves to just polytopes, though the theorem applies more broadly.
\begin{theorem}[Alexandrov–Fenchel Inequality]\th\label{thm:AFIneq}
    For polytopes \( P, Q, K_3, \dots, K_n \) in \( \mathbb{R}^n \),
    \begin{gather*}
        \MVol(P,Q, K_3, \dots, K_n)^2 \geq \MVol( P, P, K_3, \dots, K_n) \MVol( Q, Q, K_3, \dots, K_n).
    \end{gather*}
\end{theorem}
Remember that the mixed volume is just some non-negative real number, so this inequality is exactly what we are looking for in a log-concave sequence.
In fact, given any two polytopes, there's a corresponding log-concave sequence.
\begin{corollary}\th\label{thm:AFSequence}
    For any polytopes \( P, Q \subseteq \R^n \), the sequence
    \[
        \left\{ \MVol(\underbrace{P,\ldots,P}_{n-k},\underbrace{Q,\ldots,Q}_{k}) \right\}_{k=0}^{n}
    \]
    is log-concave.
\end{corollary}

This is a promising lead, but all we have is a collection of geometric definitions and a way to generate log concave sequences.
None of this actually has any clear relation to matroids.
But, just like we could make something algebraic out of the structure of a matroid, so too can we make something geometric.

\section{Bergman Fans}
Much like Chow rings, the general notion of a Bergman fan is broader than what we actually need.
Credit goes to, unsurprisingly, George Bergman, who in \cite{bergmanLogarithmicLimitsetAlgebraic1971} developed the idea of logarithmic limit-sets of algebraic varieties, which would go on to be called Bergman fans \cite{feichtnerMatroidPolytopesNested2005}.
But given we still refuse to carefully define an algebraic variety, the original presentation is not too helpful for us here.
More modern treatments, such as \cite{ardilaBergmanComplexMatroid2006,huhLogconcavityCharacteristicPolynomials2012}, have gone to show us that we can generate the Bergman fan from the combinatorial data of the matroid alone.

\subsection{Bergman Fans of Matroids}
While it may be more accurate to say we'll present the construction of a fan that can be proven to be a Bergman fan, we again simply take it as a definition.

\begin{definition}[Bergman Fan of a Matroid]\th\label{def:bergmanFan}
    Let \( \cM \) be a matroid with ground set \( E = \{ e_0, e_1, e_2, \dots, e_k \} \) and lattice of flats \( \cL \).
    Let
    \[
        \bN_E = \R^{E}/\langle e_0 + e_1 + \cdots + e_k \rangle
    \]
    be a real-valued vector space that identifies \( e_1, \dots, e_k \) with the standard basis vectors.
    For any subset \( I \subseteq E \) of the ground set, we notate
    \[
        e_I = \sum_{i \in I} e_i
    \]
    as the vector sum of each vector associated to the ground elements in \( I \).

    The \emph{Bergman fan of \( \cM \)} is a fan in \( N \) given by
    \[
        \bSigma_\cM = \left\{
        \cone\big( e_F \; | \; F \in \scrF \big)
        \;\; \middle| \;\;
        \scrF \subseteq \cL^\ast \; \text{is a flag of \(\cM\)}
        \right\}.
    \]
\end{definition}

Let's unpack this definition.
First, let's think about what space this fan lives in.
Essentially, we can think \( \bN_E = \R^{E}/\langle e_0 + e_1 + \cdots + e_k \rangle \) as \( \R^k \) where we assign all but one ground element of our matroid to the standard basis vectors.
This designates one ground element as somewhat special, it doesn't matter which one, but generically we'll call it \( e_0 \).
Then the vector associated to \( e_0 \) is the vector of all \( -1 \), as the relation in the quotient tells us
\[
    e_0 = -e_1 - e_2 - \dots - e_k.
\]

Next, let's think about the elements of our fan.
They are necessarily cones, and we see that there is one cone per flag in our matroid.
This means that there exists a \( 1 \)-dimensional cone, often called a \textit{ray}, for each proper flat, as every flat is itself a flag.
In general, the length of the flat corresponds to the dimension of the corresponding cone in the fan.
As a consequence the largest, by dimension, cones will correspond to complete flags.
Similarly, if \( F_1, F_2 \in \cL \) are non-comparable flats, then there will not be a cone generated by the rays \( e_{F_1} \) and \( e_{F_2} \).
This is how the fan structure encodes the original combinatorial data of our matroid.

A bit of notation before moving on.
We will write \( \bSigma_\cM(d) \) to be the set of all \( d \)-dimensional cones in \( \bSigma_\cM \).
We will reserve \( \rho \) to designate the rays of our Bergman fan; that is \( \rho_F \in \bSigma_\cM(1) \) for flat \( F \in \cL^\ast \).
More generally, for any flag \( \scrF = \{ F_1 \subsetneq F_2 \subsetneq \cdots \subsetneq F_\ell \} \) of our matroid, we write
\[
    \sigma_\scrF = \cone\bigl( e_{F_1}, \dots, e_{F_\ell} \bigr),
\]
for the cone  associated with \(\scrF\).

\subsubsection{An Example Bergman Fan}
As always, let's turn to our running example and see its corresponding Bergman fan.
Recall that
\[
    E = \{ a, b, c, d \} \quad \text{and} \quad \cL^\ast = \{ a, b, c, d, abd, ac, bc, cd \}.
\]

Then we can consider the space \( \bN_E = \R^E/\langle e_a + e_b + e_c + e_d \rangle\).
We will designate \( d \) as the special element and associate a basis vector of \( \R^3 \) to the remaining ground elements \( a, b, c \).
\begin{figure}[H]
    \centering
    \tdplotsetmaincoords{70}{70}
    \begin{tikzpicture}[scale=3,tdplot_main_coords]
        % \draw[draw=blue!20,fill=blue!20,fill opacity=0.5]  (0,0, 0)-- (1, 0, 0) -- (1, 1, 1) -- cycle;
        % \draw[draw=blue!20,fill=blue!20,fill opacity=0.5]  (0,0, 0)-- (0, 1, 0) -- (1, 1, 1) -- cycle;
        % \draw[draw=blue!20,fill=blue!20,fill opacity=0.5]  (0,0, 0)-- (0, 0, 1) -- (1, 1, 1) -- cycle;
        % \draw[draw=blue!20,fill=blue!20,fill opacity=0.5] (0,0, 0)-- (1, 0, 0) -- (0, -1, -1) -- cycle;
        % \draw[draw=blue!20,fill=blue!20,fill opacity=0.5] (0,0, 0)-- (0, 1, 0) -- (-1, 0, -1) -- cycle;
        % \draw[draw=blue!20,fill=blue!20,fill opacity=0.5] (0,0, 0)-- (0, 0, 1) -- (-1, -1, 0) -- cycle;
        % \draw[draw=blue!20,fill=blue!20,fill opacity=0.5] (0,0, 0)-- (-1, -1, 0) -- (-1, -1, -1) -- cycle;
        % \draw[draw=blue!20,fill=blue!20,fill opacity=0.5] (0,0, 0)-- (-1, 0, -1) -- (-1, -1, -1) -- cycle;
        % \draw[draw=blue!20,fill=blue!20,fill opacity=0.5] (0,0, 0)-- (0, -1, -1) -- (-1, -1, -1) -- cycle;
        \draw[->, color=black] (-1,0,0) -- (1.25,0,0);
        \draw[->, color=black] (0,-1,0) -- (0,1.25,0);
        \draw[->, color=black] (0,0,-.75) -- (0,0,1.25);

        \draw[->, color=blue, very thick] (0,0,0) -- (1,0,0);
        \node[right] at (1,0,0) {$e_a$};
        \draw[->,color=blue, very thick] (0,0,0) -- (0,1,0);
        \node[above] at (0,0.97,0) {$e_b$};
        \draw[->, color=blue, very thick] (0,0,0) -- (0,0,1);
        \node[left] at (0,0,1) {$e_c$};
        % \draw[->] (0,0,0) -- (1,1,1);
        % \node[right] at (1,1,1) {$\rho_{123}$};
        \draw[->, color=blue, very thick] (0,0,0) -- (-.5,-.5,-.5);
        \node[left] at (-.5,-.5,-.5) {$e_d$};
        % \draw[->] (0,0,0) -- (-1,-1,0);
        % \node[left] at (-1,-1,0) {$\rho_{03}$};
        % \draw[->,gray] (0,0,0) -- (-1,0,-1);
        % \node[left,gray] at (-1,0,-1) {$\rho_{02}$};
        % \draw[->] (0,0,0) -- (0,-1,-1);
        % \node[below] at (0,-1,-1) {$\rho_{01}$};
        % \draw[dotted] (1,1,1) -- (1,1,-1) -- (1,-1,-1) -- (1,-1,1) -- cycle;
        % \draw[dotted] (-1,1,1) -- (-1,1,-1) -- (-1,-1,-1) -- (-1,-1,1) -- cycle;
        % \draw[dotted] (1,1,1) -- (-1,1,1) -- (-1,-1,1) -- (1,-1,1) -- cycle;
        % \draw[dotted] (1,1,-1) -- (-1,1,-1) -- (-1,-1,-1) -- (1,-1,-1) -- cycle;
    \end{tikzpicture}
    \caption{The vector space \(\bN_E \) with the vectors associated to the ground set}
\end{figure}

Then we can add in all the rays of our fan, corresponding to the flats.
\begin{figure}[H]
    \centering
    \tdplotsetmaincoords{68}{55}
    \begin{tikzpicture}[scale=3,tdplot_main_coords]
        % \draw[draw=blue!20,fill=blue!20,fill opacity=0.5]  (0,0, 0)-- (1, 0, 0) -- (1, 1, 1) -- cycle;
        % \draw[draw=blue!20,fill=blue!20,fill opacity=0.5]  (0,0, 0)-- (0, 1, 0) -- (1, 1, 1) -- cycle;
        % \draw[draw=blue!20,fill=blue!20,fill opacity=0.5]  (0,0, 0)-- (0, 0, 1) -- (1, 1, 1) -- cycle;
        % \draw[draw=blue!20,fill=blue!20,fill opacity=0.5] (0,0, 0)-- (1, 0, 0) -- (0, -1, -1) -- cycle;
        % \draw[draw=blue!20,fill=blue!20,fill opacity=0.5] (0,0, 0)-- (0, 1, 0) -- (-1, 0, -1) -- cycle;
        % \draw[draw=blue!20,fill=blue!20,fill opacity=0.5] (0,0, 0)-- (0, 0, 1) -- (-1, -1, 0) -- cycle;
        % \draw[draw=blue!20,fill=blue!20,fill opacity=0.5] (0,0, 0)-- (-1, -1, 0) -- (-1, -1, -1) -- cycle;
        % \draw[draw=blue!20,fill=blue!20,fill opacity=0.5] (0,0, 0)-- (-1, 0, -1) -- (-1, -1, -1) -- cycle;
        % \draw[draw=blue!20,fill=blue!20,fill opacity=0.5] (0,0, 0)-- (0, -1, -1) -- (-1, -1, -1) -- cycle;
        \draw[->] (0,0,0) -- (1,0,0);
        \node[right](a) at (1,0,0) {$\rho_a$} ;

        \draw[->,gray] (0,0,0) -- (0,1,0);
        \node[above,gray](b) at (0,0.97,0) {$\rho_b$};

        \draw[->] (0,0,0) -- (0,0,1);
        \node[above](c) at (0,0,1) {$\rho_c$};

        \draw[->] (0,0,0) -- (-1,-1,-1);
        \node[left](d) at (-1,-1,-1) {$\rho_d$};

        \draw[->] (0,0,0) -- (0,0,-1);
        \node[right](abd) at (0,0,-1) {$\rho_{abd}$};

        \draw[->] (0,0,0) -- (1,0,1);
        \node[left](ac) at (1,0,1) {$\rho_{ac}$};

        \draw[->,gray] (0,0,0) -- (0,1,1);
        \node[left,gray](bc) at (0,1,1) {$\rho_{bc}$};

        \draw[->] (0,0,0) -- (-1,-1,0);
        \node[below](cd) at (-1,-1,0) {$\rho_{cd}$};
        %\draw[dotted] (1,1,1) -- (1,1,-1) -- (1,-1,-1) -- (1,-1,1) -- cycle;
        %\draw[dotted] (-1,1,1) -- (-1,1,-1) -- (-1,-1,-1) -- (-1,-1,1) -- cycle;
        %\draw[dotted] (1,1,1) -- (-1,1,1) -- (-1,-1,1) -- (1,-1,1) -- cycle;
        %\draw[dotted] (1,1,-1) -- (-1,1,-1) -- (-1,-1,-1) -- (1,-1,-1) -- cycle;
    \end{tikzpicture}
    \caption{The rays \( \bSigma_\cM(1)\)}
\end{figure}

Because our proper flags can only have two elements we will only have at most \( 2 \)-dimensional cones,
since we only have a cone involving the rays \(\rho_F\) and \( \rho_{F'}\) if \( F \subsetneq F' \) or \( F' \subsetneq F \).
\begin{figure}[H]
    \centering
    \tdplotsetmaincoords{28}{55}
    \begin{tikzpicture}[scale=4,tdplot_main_coords]
        \draw[->] (0,0,0) -- (1,0,0);
        \node[label=right:\(\rho_a\)](a) at (1,0,0) {};

        \draw[->,gray] (0,0,0) -- (0,1,0);
        \node[label=right:\(\rho_b\),gray](b) at (0,1,0) {};

        \draw[->] (0,0,0) -- (0,0,1);
        \node[label=above:\(\rho_c\)](c) at (0,0,1) {};

        \draw[->] (0,0,0) -- (-1,-1,-1);
        \node[label=left:\(\rho_d\)](d) at (-1,-1,-1) {};


        \draw[->] (0,0,0) -- (0,0,-1);
        \node[label=below:\(\rho_{abd}\)](abd) at (0,0,-1) {};

        \draw[->] (0,0,0) -- (1,0,1);
        \node[label=right:\(\rho_{ac}\)](ac) at (1,0,1) {};

        \draw[->,gray] (0,0,0) -- (0, 1, 1);
        \node[label=left:\(\rho_{bc}\),gray](bc) at (0, 1, 1) {};

        \draw[->] (0,0,0) -- (-1,-1,0);
        \node[label=left:\(\rho_{cd}\)](cd) at (-1,-1,0) {};

        \draw[draw=blue!20,fill=blue!20,fill opacity=0.5]  (0,0, 0)-- (1,0,0) -- (1,0,1) -- cycle; % a ac
        \draw[draw=blue!20,fill=blue!20,fill opacity=0.5]  (0,0, 0)-- (1,0,0) -- (0,0,-1) -- cycle; % a abd
        \draw[draw=blue!20,fill=blue!20,fill opacity=0.5]  (0,0, 0)-- (0, 1, 0) -- (0, 0, -1) -- cycle; % b abd
        \draw[draw=blue!20,fill=blue!20,fill opacity=0.5]  (0,0, 0)-- (0, 1, 0) -- (0, 1, 1) -- cycle; % b bc
        \draw[draw=blue!20,fill=blue!20,fill opacity=0.5] (0,0, 0)-- (0, 0, 1) -- (1, 0, 1) -- cycle; % c ac
        \draw[draw=blue!20,fill=blue!20,fill opacity=0.5] (0,0, 0)-- (0, 0, 1) -- (0, 1, 1) -- cycle; % c bc
        \draw[draw=blue!20,fill=blue!20,fill opacity=0.5] (0,0, 0)-- (0, 0, 1) -- (-1, -1, 0) -- cycle; % c cd
        \draw[draw=blue!20,fill=blue!20,fill opacity=0.5] (0,0, 0)-- (-1, -1, -1) -- (0, 0, -1) -- cycle; % d abd
        \draw[draw=blue!20,fill=blue!20,fill opacity=0.5] (0,0, 0)-- (-1, -1, -1) -- (-1, -1, 0) -- cycle; % d cd

        %\draw[dotted] (1,1,1) -- (1,1,-1) -- (1,-1,-1) -- (1,-1,1) -- cycle;
        %\draw[dotted] (-1,1,1) -- (-1,1,-1) -- (-1,-1,-1) -- (-1,-1,1) -- cycle;
        %\draw[dotted] (1,1,1) -- (-1,1,1) -- (-1,-1,1) -- (1,-1,1) -- cycle;
        %\draw[dotted] (1,1,-1) -- (-1,1,-1) -- (-1,-1,-1) -- (1,-1,-1) -- cycle;
    \end{tikzpicture}
    \caption{The Bergman fan of \( \cM \) with all of its cones}
\end{figure}

\subsection{Properties of the Bergman Fan}
Given the properties of fans we introduced earlier, it should come as no surprise that Bergman fans tend to be particularly nice.
To start, we have that Bergman fans of matroids are always both pure and simplicial.
\begin{proposition}\th\label{thm:bergFanSimPure}
    Let \( \cM \) be a matroid of rank \( r + 1 \).
    The associated Bergman fan \( \bSigma_\cM \) is a simplicial \( r \)-fan.
\end{proposition}
\begin{proof}
    To prove both of these properties we need one simple fact.
    By definition, a cone \( \sigma \in \bSigma \) is of the form
    \[
        \sigma = \cone( e_{F_1}, \dots, e_{F_k} ),
    \]
    for \( F_1 \subsetneq \cdots \subsetneq F_k \) some flag of \( \cM \).
    What we need is that \( \{e_{F_1}, e_{F_2}, \dots, e_{F_k} \} \) is linearly independent.
    To see this, consider the equation
    \[
        \lambda_{1}e_{F_1} + \cdots + \lambda_{k}e_{F_{k}} = 0.
    \]
    Recall that we defined \( e_{F_j} = \sum_{i \in F_{j} } e_i \) to be the sum of the vectors associated to the ground set, so we can rewrite our equation as
    \[
        \lambda_{1}\sum_{i \in F_{1} } e_i + \cdots + \lambda_{k}\sum_{i \in F_{k} } e_i = 0.
    \]
    But now, given that our flats form a flag that have strict inclusion, we may reorder terms to get
    \[
        (\lambda_1 + \lambda_2 + \cdots + \lambda_k) \sum_{\mathclap{i \in F_1}} e_i
        + (\lambda_2 + \cdots + \lambda_k) \sum_{\mathclap{i \in F_2 \setminus F_1}} e_i
        + \cdots
        + \lambda_k \sum_{\mathclap{i \in F_k \setminus F_{k-1}}} e_i  = 0.
    \]
    Since each of these sums involves a disjoint set of the vectors associated to the ground set, and any proper subset of these ground set vectors is linearly independent, it must be that \( \lambda_1 = \cdots = \lambda_k = 0 \), thus showing the set \( \{e_{F_1}, e_{F_2}, \dots, e_{F_k} \} \) is linearly independent.

    With this we immediately have that \( \bSigma_\cM \) is simplicial.
    For any cone \( \sigma_\scrF \in \bSigma_\cM \) associated the flag \( \scrF \), it is generated by \( |\scrF| \) vectors, and because those vectors are linearly independent \( \dim(\sigma_\scrF) = |\scrF| \).

    Now, recall that for a \( d \)-dimensional simplical cone with generating rays \(  V = \{ v_1, \dots, v_d \} \), for any subset \( V' \subseteq V \),
    \( \cone(V') \) is a face of \( \cone(V) \).
    Since we've  shown that every cone of \( \bSigma_\cM \) is simplicial, we have that any cone associated to a non-complete flag is non-maximal.
    This follows from the fact that if \( \scrF \) is not a complete flag then there exists another flag \( \scrF' \) such that \( \scrF \subsetneq \scrF' \).
    Then we have that \( \sigma_\scrF(1) \subsetneq \sigma_{\scrF'}(1) \) and so \( \sigma_\scrF \) is a face of \( \sigma_{\scrF'} \) and by definition non-maximal.
    Thus, every maximal cone of \( \bSigma_\cM \) is associated to a complete flag, and any complete flag will have \( r \) elements.
    From above, we may conclude that every maximal cone has dimension \( r \).

    With that we've shown that \( \bSigma_\cM \) is a simplicial \( r \)-fan.

\end{proof}
Now that we know that a Bergman fan must necessarily be simplicial and pure, we can go on to show that the Bergman fan is not only tropical but also balanced.
\begin{proposition}
    Let \( \cM \) be a matroid.
    The Bergman fan \( \bSigma_\cM \) is a balanced tropical fan.
\end{proposition}
\begin{proof}
    Let \( \cM = (E, \cL) \) be a matroid of rank \( r + 1 \) and \( \bSigma_\cM \subseteq \bN_E \) its Bergman fan.
    We'll mark the rays \( \rho \in \bSigma_\cM \) such that \( u_\rho = e_{F_\rho} \) where \( F_\rho \in \cL \) is the flat associated to \( \rho \), and we continue to use the convention that
    \[
        e_{I} = \sum_{i \in I} e_i,
    \]
    for any \( I \subseteq E \).
    Recall that for \( \bSigma_\cM \) to be balanced it must meet the balancing condition
    \[
        \sum_{\mathclap{\substack{\sigma \in \bSigma(r) \\ \tau \preceq \sigma}}} u_{\sigma \setminus \tau} \in \Span(\tau),
    \]
    for every \( \tau \in \bSigma(r-1) \).

    Let us fix an arbitrary \( \tau \in \bSigma(r-1) \) and consider the flag of flats associated to it
    \[
        \scrF = \{ F_1 \subsetneq \cdots \subsetneq F_{k-1} \subsetneq  F_{k+1} \subsetneq \cdots \subsetneq F_{r} \}.
    \]
    That is, \( \scrF \) is a flag of length \( r - 1 \) with just a single rank \( k \) flat missing from being a complete flag.
    Any maximal cone \( \sigma \) that contains \( \tau \) as a face will be associated to the same set of flats but with the rank \( k \) flat, \( F_{k} \), added.
    We can now rephrase our balancing condition in these terms; we wish to show
    \[
        \sum_{\mathclap{\substack{F_k \in \cL \\ F_{k-1} \subsetneq F_k \subsetneq F_{k+1} }}} e_{F_{k}} \in \Span(e_{F_1}, \dots, e_{F_{k-1}}, e_{F_{k+1}}, \dots, e_{F_r}).
    \]
    % We'll split this proof into two cases, based on the rank of the missing flat.

    % In the first let us assume that \( \scrF \) does not have a rank 1 flat, so
    % \[
    %     \scrF = \{ F_2 \subsetneq \cdots \subsetneq F_d \}.
    % \]
    % Recall that property \textit{iii.} of \th\ref{thm:simpMatroidProps} tells us that if \( F \) is a flat then for each ground element \( i \in F \) the rank 1 flat \( F_i \) is a subset of \( F \).
    % And so this gives us
    % \begin{align*}
    %     \sum_{\mathclap{\substack{F_1 \in \cL                    \\ F_1 \subsetneq F_{2} }}} e_{F_{1}}
    %      & = \sum_{\mathclap{\substack{i \in F_{2} }}} e_{F_{i}} \\
    %      & = \sum_{\mathclap{\substack{i \in F_{2} }}} e_i       \\
    %      & = e_{F_2}.
    % \end{align*}
    % Since \( e_{F_2} \) is clearly in the span of \( \{ e_{F_2}, \dots, e_{F_r} \} \), our fan is balanced in this case.

    For convenience, let us call \( F_0 = \emptyset \) and \( F_{r+1} = E \), as they are the unique rank 0 and \( r \) flats, respectively.
    We'll set the convention that \( e_{F_0} = \sum_{i \in \emptyset} e_i = 0 \).
    Similarly, we recall that \( e_E = 0 \).
    This means we can equivalently show that,
    \[
        \sum_{\mathclap{\substack{F_k \in \cL \\ F_{k-1} \subsetneq F_k \subsetneq F_{k+1} }}} e_{F_{k}} \in \Span( e_{F_0}, e_{F_1}, \dots, e_{k-1}, e_{k+1}, \dots, e_{F_r}, e_{F_{r+1}} ).
    \]
    Let's first use our basic definitions to rewrite our sum
    \begin{align*}
        \sum_{\mathclap{\substack{F_k \in \cL \\ F_{k-1} \subsetneq F_k \subsetneq F_{k+1} }}} e_{F_{k}}
        \;\; & = \;\;\;
        \sum_{\mathclap{\substack{F_k \in \cL \\ F_{k-1} \subsetneq F_k \subsetneq F_{k+1} }}}\, \left( \, \sum_{\mathclap{ i \in F_k }}e_i \right) \\
             & = \;\;\;
        \sum_{\mathclap{\substack{F_k \in \cL \\ F_{k-1} \subsetneq F_k \subsetneq F_{k+1} }}}\, \left( \; \sum_{\mathclap{ i \in F_{k-1} }}e_i + \sum_{\mathclap{ i \in F_{k} \setminus F_{k-1} }}e_i \, \right) \\
             & = \;\;\;
        \sum_{\mathclap{\substack{F_k \in \cL \\ F_{k-1} \subsetneq F_k \subsetneq F_{k+1} }}}\, \left( e_{F_{k-1}} + e_{F_k \setminus F_{k-1}} \right).
        % \sum_{\mathclap{\substack{F_k \in \cL \\ F_{k-1} \subsetneq F_k \subsetneq F_{k+1} }}}\, \left( e_{F_{k-1}} + \sum_{\mathclap{ i \in F_{k} \setminus F_{k-1} }}e_i \, \right).
    \end{align*}
    Indexing nightmare aside, we note that the term \( e_{F_{k-1}}\) in the sum is indepentent of choice of \( F_k \).
    We'll let \( \Lambda > 0 \) be the number of complete flags that contain \( \scrF \), letting us write
    \[
        \sum_{\mathclap{\substack{F_k \in \cL \\ F_{k-1} \subsetneq F_k \subsetneq F_{k+1} }}}\, \left( e_{F_{k-1}} + e_{F_k \setminus F_{k-1}} \right)
        = \Lambda e_{F_{k-1}} + \sum_{\mathclap{\substack{F_k \in \cL \\ F_{k-1} \subsetneq F_k \subsetneq F_{k+1} }}}\, \left( e_{F_k \setminus F_{k-1}} \right).
    \]

    Now, if we have flats \( F_{k-1} \subsetneq F_{k+1} \) of rank \( k-1 \) and \( k+1 \) respectively, consider an element of the ground set \( i \in F_{k+1} \setminus F_{k-1} \).
    By definition, the closure \( F = \cl(F_{k-1} \cup i) \) is a flat.
    Since \( i \) was not in \( F_{k-1} \), \( \rk(F) > \rk(F_{k-1}) \), and in particular must be rank \( k \) since adding a single element can only increase the rank by at most one.
    By property~\ref{def:F3} of the properties of flats, we know \( F \) is the unique flat such that \( F_{k-1} \subsetneq F \) and \( i \in F \), so we have that \( F_{k-1} \subsetneq F \subsetneq F_{k+1} \).
    Further, the same property guarantees us that each \( i \in F_{k+1} \setminus F_{k-1} \) will appear in exactly one rank \( k \) flat, \( F_{k,i} \) such that \( F_{k-1} \subsetneq F_{k,i} \subsetneq F_{k+1} \).
    This face lets us write
    \begin{align*}
        \Lambda e_{F_{k-1}} + \sum_{\mathclap{\substack{F_k \in \cL                                 \\ F_{k-1} \subsetneq F_k \subsetneq F_{k+1} }}}\, \left( e_{F_k \setminus F_{k-1}} \right)
         & = \Lambda e_{F_{k-1}} + e_{F_{k+1} \setminus F_{k-1}}                                    \\
         & = (\Lambda - 1) e_{F_{k-1}} + \left( e_{F_{k-1}} + e_{F_{k+1} \setminus F_{k-1}} \right) \\
         & = (\Lambda - 1) e_{F_{k-1}} + e_{F_{k+1}}.
    \end{align*}
    In this form we can clearly see this is in the span of \( \{ e_{F_0}, e_{F_1}, \dots, e_{k-1}, e_{k+1}, \dots, e_{F_r}, e_{F_{r+1}} \} \).
    With this, we've shown the Bergman fan of \( \cM \) is balanced.
\end{proof}
Many of the following theorems are about tropical fans more generally, and as such the weight function plays a role.
However, since all Bergman fans of matroids are balanced, we will state the theorems just for balanced fans and remove refrences to the weight function.
Simplifying all our lives a little bit.

% While we still don't want to go into the deep details of tropical geometry, we still think it is worthwhile to note that if \( \bSigma \) is a tropical fan, then it has an associated Chow ring \( A^\bullet(\bSigma) \).
% This is built somewhat analogously to how we developed the Chow ring of a matroid,  starting with a polynomial ring with one variable per ray in \( \bSigma(1) \) and using a quotient to induce relations among the rays based on the cones of the fan.
% Notably, if \( \bSigma_{\cM} \) is the Bergman fan of a matroid \( \cM \), then \( A^\bullet(\bSigma_\cM) = A^\bullet(\cM) \).

We now have a geometric object associated with our matroid, but we still don't yet have a bridge back to the realm of algebra as promised.
Also, we seemed to hint volume was going to be useful, but there's no obvious way to take the volume of a fan.
To wrap this up, we move on to our final geometric object.

\section{Normal Complexes}
The work in this section is by far the most recent, coming from work within the last two years at time of writing.
We will provide a brief summary of the work by Nathanson and Ross \cite{nathansonTropicalFansNormal2023} and by Nowak and Ross, jointly with the author \cite{nowakMixedVolumesNormal2023}.

This section will use the Bergman fan of a matroid to make an object called a normal complex.
From this we can develop a notion of volume, as well as present theorems that relate this volume back to the Chow ring, finally giving us all the components necessary for our main result.

\subsection{The Normal Complex of a Fan}
In essence, a normal complex of a fan is simply a truncation of a fan into a polytopal complex using hyperplanes normal to the rays of the fan, thus the name.
They were initially developed in \cite{nathansonTropicalFansNormal2023}, and the following definitions and propositions come from this work.
However, we can't just take any truncation of our fan.
We need a way to specify the truncations that will work, and so we introduce the idea of cubical and pseudocubical values.

\begin{definition}[Cubical Values]\th\label{def:cubical}
    Let \( \bSigma \subseteq \R^n \) be a marked, simplicial \( d \)-fan and \( \ast \in \inn(\R^n) \) an inner product.
    Pick a vector \( z \in \R^{\bSigma(1)} \) that associates a real number to each ray of our fan.
    For each ray \( \rho \in \bSigma(1) \) we have a corresponding hyperplane and half-space
    \[
        \HP_{\rho, \ast}(z) = \{ v \in N \; | \; v \ast u_\rho = z \}
        \quad \text{and} \quad
        \LH_{\rho, \ast}(z) = \{ v \in N \; | \; v \ast u_\rho \leq z \}.
    \]

    For each cone \( \sigma \in \bSigma \), let \( w_{\sigma,\ast}(z) \) be the unique value such that
    \[
        w_{\sigma}(z) \ast u_\rho = z_\rho
    \]
    for each ray \(\rho \in \sigma(1)\).

    We say \( z \) is \emph{cubical} if for all \( \sigma \in \bSigma \),
    \[
        w_{\sigma}(z)  \in \sigma^\circ,
    \]
    where \( \sigma^\circ \) is the relative interior of \( \sigma \).
\end{definition}
\begin{definition}[Pseudocubical]\th\label{def:pseudocubical}
    Given everything exactly as in the previous definition, if we instead require only that
    \[
        w_{\sigma}(z) \in \sigma
    \]
    for all \( \sigma \in \bSigma \), we say \( z \) is \emph{pseudocubical}.
\end{definition}
For a given fan \( \bSigma \subseteq N \) and inner product \( \ast \in \inn(N) \), we denote cubical values as \( \Cub(\bSigma, \ast) \subseteq \R^{\bSigma(1)} \) and the set of pseudocubical values as \( \pCub(\bSigma, \ast) \subseteq \R^{\bSigma(1)} \).
This is all to say that when we select values to generate truncating hyperplanes for each ray, we want the intersections of the hyperplanes to lie within the cones of the fan.
We can see an example in \( 2 \)-dimensions.
\begin{figure}[H]
    \centering

    \begin{tikzpicture}[scale=3]
        \draw[draw=blue!10,fill=blue!10, fill opacity=.5]    (.1,.04) -- (.95,.04) -- (.95,.89);
        \draw[thick,fill=green!20, fill opacity=.5] (0,0) -- (0.8,0) -- (0.8,0.4) -- (0.6,0.6);
        \draw[->] (0,0) -- (1,0);
        \draw[->] (0,0) -- (1,1);
        \node[right] at (1,0) {$\rho_1$};
        \node[right] at (1,1) {$\rho_{2}$};
        \draw[thick] (0.8,-0.1) -- (0.8,0.55);
        \draw[thick] (0.4,0.8) -- (0.9,0.3);
        \node[] at (0.8,0.4) {$\bullet$};
        \node[] at (0.5,-.2) {cubical};
        %
        \draw[draw=blue!10,fill=blue!10, fill opacity=.5]    (2.1,.04) -- (2.95,.04) -- (2.95,.89);
        \draw[thick,fill=green!20, fill opacity=.5] (2,0) -- (2.6,0) -- (2.6,0.6);
        \draw[->] (2,0) -- (3,0);
        \draw[->] (2,0) -- (3,1);
        \node[right] at (3,0) {$\rho_1$};
        \node[right] at (3,1) {$\rho_{2}$};
        \draw[thick] (2.6,-0.1) -- (2.6,0.8);
        \draw[thick] (2.4,0.8) -- (2.9,0.3);
        \node[] at (2.6,0.6) {$\bullet$};
        \node[] at (2.5,-.2) {pseudocubical};
        %
        \draw[draw=blue!10,fill=blue!10, fill opacity=.5]    (4.1,.04) -- (4.95,.04) -- (4.95,.89);
        \draw[thick,fill=green!20, fill opacity=.5] (4,0) -- (4.5,0) -- (4.5,0.5);
        \draw[->] (4,0) -- (5,0);
        \draw[->] (4,0) -- (5,1);
        \node[right] at (5,0) {$\rho_1$};
        \node[right] at (5,1) {$\rho_{2}$};
        \draw[thick] (4.5,-0.1) -- (4.5,0.9);
        \draw[thick] (4.35,0.85) -- (4.9,0.3);
        \node[] at (4.5,0.7) {$\bullet$};
        \node[] at (4.5,-.2) {not pseudocubical};
    \end{tikzpicture}
    \caption{Examples of possible normal truncating hyperplane arrangements for a simple \( 2 \)-dimensional fan}
\end{figure}
With this, we can now define a normal complex.
\begin{definition}[Normal Complex]\th\label{def:normalComplex}
    Let \( \bSigma \subseteq N \) be a simplicial \( d \)-fan, with marked point \( u_\rho \) on each ray \( \rho \in \bSigma(1) \).
    Choose an inner product \( \ast \in \inn(N) \) and a pseudocubical vector \( z \in \pCub(\bSigma, \ast) \).
    Recall that for each ray \( \rho \in \bSigma(1) \) we have a corresponding hyperplane and half-space
    \[
        \HP_{\rho, \ast}(z) = \{ v \in N \; | \; v \ast u_\rho = z \}
        \quad \text{and} \quad
        \LH_{\rho, \ast}(z) = \{ v \in N \; | \; v \ast u_\rho \leq z \}.
    \]
    For each cone \( \sigma \in \bSigma \), we define a polytope \( P_{\sigma,\ast}(z) \) given by
    \[
        P_{\sigma,\ast}(z) = \sigma \cap \left( \bigcap_{\rho \in \sigma(1)} \LH_{\rho, \ast}(z)\right).
    \]

    The \emph{normal complex} of \( \bSigma \) is the polytopal complex
    \[
        C_{\bSigma, \ast}(z) = \bigcup_{\sigma \in \bSigma} P_{\sigma, \ast} (z)
    \]
\end{definition}
That these truncations of fans give us well-defined polytopal complex is a result of Proposition~3.7 in \cite{nathansonTropicalFansNormal2023}.
The basic idea is that because the pseudocublical condition insures the hyperplanes intersect within each cone exactly at one point, these intersection points define the vertices of each polytope in the complex.
Those with more knowledge of polytopes can also see there is hyperplane description of these polytopes readily available as well.
Here is another place where an example is worth many words.
% Let's start with one in \( 2 \)-dimensions.
\begin{figure}[H]
    \begin{subfigure}[t]{0.48\textwidth}
        \centering
        \begin{tikzpicture}[scale=2.7]
            \draw[draw=blue!10,fill=blue!10, fill opacity=.5]    (.17,.07) -- (.9,.07) -- (.9,.8);
            % \draw[draw=blue!10,fill=blue!10, fill opacity=.5]    (.07,-.07) -- (.9,-.07) -- (.9,-.9) -- (.07,-.9);
            \draw[draw=blue!10,fill=blue!10, fill opacity=.5]    (-.07,-.17) -- (-.07,-.9) -- (-.8,-.9);
            \draw[draw=blue!10,fill=blue!10, fill opacity=.5 ]    (-.17,-.07) -- (-.9,-.07) -- (-.9,-.8);
            % \draw[draw=blue!10,fill=blue!10, fill opacity=.5]    (-.07,.07) -- (-.9,.07) -- (-.9,.9) -- (-.07,.9);
            % \draw[draw=blue!10,fill=blue!10, fill opacity=.5]    (.07,.2) -- (.07,.9) -- (.8,.9);
            \draw[->] (0,0) -- (1,0);
            \draw[->] (0,0) -- (1,1);
            % \draw[->] (0,0) -- (0,1);
            \draw[->] (0,0) -- (-1,0);
            \draw[->] (0,0) -- (-1,-1);
            \draw[->] (0,0) -- (0,-1);
            % \node[right] at (1,0) {$\rho_1$};
            % \node[right] at (1,1) {$\rho_{12}$};
            % \node[right] at (0,1) {$\rho_2$};
            % \node[left] at (-1,0) {$\rho_{02}$};
            % \node[right] at (-1,-1) {$\rho_0$};
            % \node[right] at (0,-1) {$\rho_{01}$};
            % \draw[thick,color=Red] (-1,.8) -- (.6,.8);
            \draw[thick,color=Red] (-.6,-.8) -- (.4,-.8);
            \draw[thick,color=Red] (-.8,-.6) -- (-.8,.4);
            \draw[thick,color=Red] (.85,-.4) -- (.85,.6);
            \draw[thick,color=Red] (0,1.2) -- (1.2,0);
            \draw[thick,color=Red] (0,-1.2) -- (-1.2,0);
        \end{tikzpicture}
        \subcaption{A fan with a possible arrangement of  hyperplanes given by a cubical value}
    \end{subfigure}
    \hfill
    \begin{subfigure}[t]{0.48\textwidth}
        \centering
        \begin{tikzpicture}[scale=1.5]
            \draw[thick,fill=green!20, fill opacity=.5] (0,0) -- (2.5, 0) -- (2.5,.5)  -- (1.5, 1.5) -- cycle;
            \draw[thick,fill=green!20, fill opacity=.5] (0,0) -- (0, -2) -- (-1,-2) -- (-2, -1) -- (-2,0) -- cycle ;
            % \draw[thick,fill=green!20, fill opacity=.5] (1,2) -- (2.5,.5) -- (2.5,-2) -- (-1,-2) -- (-2,-1) -- (-2,2) -- (1,2);

            \draw[thick] (0,0) -- (2.5,0);
            \draw[thick] (0,0) -- (1.5,1.5);
            % \draw[thick] (0,0) -- (0,2);
            \draw[thick] (0,0) -- (-2,0);
            \draw[thick] (0,0) -- (-1.5,-1.5);
            \draw[thick] (0,0) -- (0,-2);
        \end{tikzpicture}
        \caption{The resulting normal complex of the fan given the choice of cubical value}
    \end{subfigure}
\end{figure}
% We've already seen the Bergman fan of our example matroid, for which we can now give an example of a normal complex.
% \begin{figure}[H]
%     \begin{subfigure}[t]{0.48\textwidth}
%         \centering
%         \tdplotsetmaincoords{68}{55}
%         \begin{tikzpicture}[scale=2.8,tdplot_main_coords]
%             \draw[draw=blue!20,fill=blue!20,fill opacity=0.5]  (0,0, 0)-- (1, 0, 0) -- (1, 1, 1) -- cycle;
%             \draw[draw=blue!20,fill=blue!20,fill opacity=0.5]  (0,0, 0)-- (0, 1, 0) -- (1, 1, 1) -- cycle;
%             \draw[draw=blue!20,fill=blue!20,fill opacity=0.5]  (0,0, 0)-- (0, 0, 1) -- (1, 1, 1) -- cycle;
%             \draw[draw=blue!20,fill=blue!20,fill opacity=0.5] (0,0, 0)-- (1, 0, 0) -- (0, -1, -1) -- cycle;
%             \draw[draw=blue!20,fill=blue!20,fill opacity=0.5] (0,0, 0)-- (0, 1, 0) -- (-1, 0, -1) -- cycle;
%             \draw[draw=blue!20,fill=blue!20,fill opacity=0.5] (0,0, 0)-- (0, 0, 1) -- (-1, -1, 0) -- cycle;
%             \draw[draw=blue!20,fill=blue!20,fill opacity=0.5] (0,0, 0)-- (-1, -1, 0) -- (-1, -1, -1) -- cycle;
%             \draw[draw=blue!20,fill=blue!20,fill opacity=0.5] (0,0, 0)-- (-1, 0, -1) -- (-1, -1, -1) -- cycle;
%             \draw[draw=blue!20,fill=blue!20,fill opacity=0.5] (0,0, 0)-- (0, -1, -1) -- (-1, -1, -1) -- cycle;
%             \draw[->] (0,0,0) -- (1,0,0);
%             % \node[right] at (1,0,0) {$\rho_1$};
%             \draw[->,gray] (0,0,0) -- (0,1,0);
%             % \node[above,gray] at (0,0.97,0) {$\rho_2$};
%             \draw[->] (0,0,0) -- (0,0,1);
%             % \node[above] at (0,0,1) {$\rho_3$};
%             \draw[->] (0,0,0) -- (1,1,1);
%             % \node[right] at (1,1,1) {$\rho_{123}$};
%             \draw[->] (0,0,0) -- (-1,-1,-1);
%             % \node[left] at (-1,-1,-1) {$\rho_0$};
%             \draw[->] (0,0,0) -- (-1,-1,0);
%             % \node[left] at (-1,-1,0) {$\rho_{03}$};
%             \draw[->,gray] (0,0,0) -- (-1,0,-1);
%             % \node[left,gray] at (-1,0,-1) {$\rho_{02}$};
%             \draw[->] (0,0,0) -- (0,-1,-1);
%             % \node[below] at (0,-1,-1) {$\rho_{01}$};
%             %\draw[dotted] (1,1,1) -- (1,1,-1) -- (1,-1,-1) -- (1,-1,1) -- cycle; 
%             %\draw[dotted] (-1,1,1) -- (-1,1,-1) -- (-1,-1,-1) -- (-1,-1,1) -- cycle; 
%             %\draw[dotted] (1,1,1) -- (-1,1,1) -- (-1,-1,1) -- (1,-1,1) -- cycle; 
%             %\draw[dotted] (1,1,-1) -- (-1,1,-1) -- (-1,-1,-1) -- (1,-1,-1) -- cycle; 
%         \end{tikzpicture}
%         \subcaption*{\(\bSigma_\cM\)}
%     \end{subfigure}
%     \hfill
%     \begin{subfigure}[t]{0.48\textwidth}
%         \centering
%         \tdplotsetmaincoords{68}{55}
%         \begin{tikzpicture}[scale=1.5,tdplot_main_coords]

%             \draw[draw=green!20, fill=green!20, fill opacity=.5] (0,0,0) -- (0, 0, 1.6) -- (1, 1, 1.6) -- (1.2, 1.2, 1.2) -- cycle; %d,bcd
%             \draw[draw=green!20, fill=green!20, fill opacity=.5] (0,0,0) -- (0, 1.6, 0) -- (1, 1.6, 1) -- (1.2, 1.2, 1.2) -- cycle; %b,bcd
%             \draw[draw=green!20, fill=green!20, fill opacity=.5] (0,0,0) -- (1.6, 0, 0) -- (1.6, 1, 1) -- (1.2, 1.2, 1.2) -- cycle; %c,bcd
%             \draw[draw=green!20, fill=green!20, fill opacity=.5] (0,0,0) -- (0, 0, 1.6) -- (-1.6, -1.6, 1.6) -- (-1.6, -1.6, 0) -- cycle; %d,ad
%             \draw[draw=green!20, fill=green!20, fill opacity=.5] (0,0,0) -- (0, 1.6, 0) -- (-1.6, 1.6, -1.6) -- (-1.6, 0, -1.6) -- cycle; %b,ab
%             \draw[draw=green!20, fill=green!20, fill opacity=.5] (0,0,0) -- (1.6, 0, 0) -- (1.6, -1.6, -1.6) -- (0, -1.6, -1.6) -- cycle; %c,ac
%             \draw[draw=green!20, fill=green!20, fill opacity=.5] (0,0,0) -- (-1.6, -1.6, 0) -- (-1.6, -1.6, -.4) -- (-1.2, -1.2, -1.2) -- cycle; %a,ad
%             \draw[draw=green!20, fill=green!20, fill opacity=.5] (0,0,0) -- (-1.6, 0, -1.6) -- (-1.6, -.4, -1.6) -- (-1.2, -1.2, -1.2) -- cycle; %a,ac
%             \draw[draw=green!20, fill=green!20, fill opacity=.5] (0,0,0) -- (0, -1.6, -1.6) -- (-.4, -1.6, -1.6) -- (-1.2, -1.2, -1.2) -- cycle; %a,ab

%             \draw[thick] (0, 0, 1.6) -- (1, 1, 1.6) -- (1.2, 1.2, 1.2);
%             \draw[dashed] (0, 1.6, 0) -- (1, 1.6, 1) -- (1.2, 1.2, 1.2);
%             \draw[thick] (.7, 1.6, .7) -- (1, 1.6, 1) -- (1.2, 1.2, 1.2);
%             \draw[thick] (1.6, 0, 0) -- (1.6, 1, 1) -- (1.2, 1.2, 1.2);
%             \draw[thick] (0, 0, 1.6)  -- (-1.6,  -1.6, 1.6)  -- (-1.6, -1.6, 0);
%             \draw[dashed] (0, 1.6, 0) -- (-1.6,  1.6, -1.6) -- (-1.6, 0, -1.6);
%             \draw[thick] (1.6, 0, 0) -- (1.6, -1.6, -1.6) -- (0, -1.6, -1.6);
%             \draw[thick] (-1.6, -1.6, 0) -- (-1.6, -1.6, -.4) -- (-1.2, -1.2, -1.2);
%             \draw[dashed] (-1.6, 0, -1.6) -- (-1.6, -.4, -1.6) -- (-1.2, -1.2, -1.2);
%             \draw[thick] (0, -1.6, -1.6) -- (-.4, -1.6, -1.6) -- (-1.2, -1.2, -1.2);

%             \draw[thick] (0, 0, 0) -- (1.2, 1.2, 1.2);
%             \draw[thick] (0, 0, 0) -- (0, 0, 1.6);
%             \draw[dashed] (0, 0, 0) -- (0, 1.6, 0);
%             \draw[thick] (0, 0, 0) -- (1.6, 0, 0);
%             \draw[thick] (0, 0, 0) -- (0, -1.6, -1.6);
%             \draw[dashed] (0, 0, 0) -- (-1.6, 0, -1.6);
%             \draw[thick] (0, 0, 0) -- (-1.6, -1.6, 0);
%             \draw[thick] (0, 0, 0) -- (-1.2, -1.2, -1.2);

%             %\draw[gray,->] (0,0,0) -- (1,0,0);
%             %\draw[gray,->] (0,0,0) -- (0,1,0);
%             %\draw[gray,->] (0,0,0) -- (0,0,1);
%             %\draw[gray,->] (0,0,0) -- (1,1,1);
%             %\draw[gray,->] (0,0,0) -- (-1,-1,-1);
%             %\draw[gray,->] (0,0,0) -- (-1,-1,0);
%             %\draw[gray,->] (0,0,0) -- (-1,0,-1);
%             %\draw[gray,->] (0,0,0) -- (0,-1,-1);
%         \end{tikzpicture}
%         \subcaption*{\( C_{\bSigma_\cM, \ast}(z) \)}
%     \end{subfigure}
%     \caption{The Bergman fan and a normal complex of our example matroid}
%     \todo{My example matroid is slightly different. I need to make a corresponging normal complex for it}
% \end{figure}

\subsection{The Volume of a Normal Complex}
After so long hinting at the importance of volume, we finally have something related to our matroid we can actually take the volume of.
Remember that we, rather lazily, only defined volumes on polytopes.
We can extend the definition to normal complexes, though out of necessity how we do so is a bit fussy.
\begin{definition}[Volume of a Normal Complex]\th\label{def:volNormalComplex}
    Let \( \bN \) be a real-valued vector space and \( \bSigma \subseteq \bN \) be a simplicial \( d \)-fan with marked points \( u_\rho \).
    Choose an inner product \( \ast \in \inn(\bN) \) and define
    \[
        \bN_\sigma = \Span\big(u_\rho \; | \; \rho \in \sigma(1)\big) \subseteq \bN.
    \]
    Then, let \( \bM_\sigma \) be the vector space dual to \( \bN_\sigma \).
    Using \( \{ u_\rho \; | \; \rho \in \sigma(1) \} \) as the basis of \( \bN_\sigma \), and using our chosen inner product, we may identify the dual basis \( \{ u^\rho \; | \; \rho \in \sigma(1) \} \) of \( \bM_\sigma \) as a subset of \( \bN_\sigma \).
    For each \( \sigma \in \bSigma(d) \), we choose the volume function
    \[
        \Vol_\sigma \,:\; \{\text{Polytopes in \( \bN_\sigma \)} \}  \to \R_{\geq 0}
    \]
    characterized by \( \Vol_\sigma\big( \conv( \{0\} \cup \{u^\rho \; | \; \rho \in \sigma(1)\})\big) = 1 \).

    For pseudocubical value \( z \in \pCub(\sigma, \ast) \), the \emph{volume of the normal complex} \(  C_{\bSigma, \ast}(z) \) is
    \[
        \Vol_{\bSigma, \ast}(z) = \sum_{\sigma \in \bSigma(d)} \Vol_\sigma\big(P_{\sigma, \ast}(z)\big).
    \]
\end{definition}

The basic idea here is that we are taking the volume of each top-dimensional polytope in the complex and summing them.
Nothing too wild there.
Most of the verbosity in the definition comes from the fact that we have to choose the volume function for each component rather carefully.
The specifics of why this must be done we leave to the reader to explore in \cite{nathansonTropicalFansNormal2023}, but suffice to say it is necessary for the main theorem of that paper.
This theorem provides the first part of the bridge we are developing between geometry and algebra.
\begin{theorem}[Nathanson-Ross 2021]\th\label{thm:volDegCorrespondence}
    Let \( \bSigma \) be a balanced \( d \)-fan.
    Choose an inner product \( \ast \in \inn(N) \) and pseudocubical value \( z  \in \pCub( \bSigma, \ast ) \).
    We define
    \[
        D(z) = \sum_{\rho \in \bSigma(1)} z_\rho x_\rho \in A^1(\bSigma),
    \]
    a divisor of the Chow ring.
    Then
    \[
        \Vol_{\bSigma, \ast}(z) = \deg\big( D(z)^d \big).
    \]
\end{theorem}

Finally, we have the first link back to the Chow ring.
By carefully taking volumes of the normal complex, we can evaluate top degrees of divisors under the degree map.
While this is very cool, we have a problem.
This only allows us to evaluate a single divisor raised to the top power under the degree map.
Recall that we want to reason about elements of the form \( \alpha^{d-k}\beta^k \).
These are called mixed degrees of divisors, and perhaps that is a good hint as to what we need to develop next.

\subsection{Mixed Volumes of Normal Complexes}
Here we finally can justify introducing the mixed volume function earlier.
We can't define the mixed volume of the full normal complexes directly, as the Minkowski sum of two normal complexes would certainly not be a normal complex itself in general.
But like volume we could define it component wise.

A distinction from the original mixed volume function is that here is that we can't take the mixed volume of an arbitrary collection of normal complexes.
We can only take the mixed volumes of normal complexes that have the same underlying fan.

\begin{definition}[Mixed Volume of Normal Complexes]\th\label{def:mixedVolNormalComplex}
    Let \( \bSigma \subseteq N \) be a simplicial \( d \)-fan, \( \ast \in \inn(N) \) be an inner product, and pseudocubical values \( z_1, \dots, z_d \in \pCub(\bSigma, \ast)\).
    The mixed volume of \( C_{\bSigma, \ast}(z_1), \dots, C_{\bSigma, \ast}(z_d) \), written \( \MVol_{\bSigma, \ast}(z_1, \dots, z_d ) \) is given by
    \[
        \MVol_{\bSigma, \ast}(z_1, \dots, z_d ) = \sum_{\sigma \in \bSigma(d)} \MVol_\sigma\big(P_{\sigma, \ast}(z_1), \dots, P_{\sigma, \ast}(z_d)\big),
    \]
    where \( \MVol_\sigma \) is the mixed volume of polytopes as defined above using the volume function \( \Vol_\sigma \).
\end{definition}
Here, the basic idea as to why this may work is that taking the Minkowski sum of polytopes in the same cone will produce another polytope in this cone.
\begin{figure}[H]
    \centering
    \includegraphics[width=1.05\textwidth]{./images/mvol_ex}
    \caption{Component wise Minkowski sums of two normal complexes of the same fan; note each polytope in the sum is still in its correct cone}
\end{figure}
That this works in general is not obvious.
Nor is it immediate that this newly defined function has all the nice properties of the original mixed volumes.
The development and proof of the following proposition was originally the work of Lauren Nowak in her master's thesis~\cite{nowakMixedVolumesNormal2022}.
This work was then incorperated into~\cite{nowakMixedVolumesNormal2023} from which Proposition~3.1 gives us the following guarantee.
\begin{proposition}
    Let \( \bSigma \subset N \) be a simplicial \( d \)-fan, \( \ast \in \inn(N) \) an inner product on \( \R^n \), and  pseudocubical values \( z_1, \dots, z_n \in \pCub(\bSigma, \ast)\).

    The function
    \[
        \MVol_{\bSigma, \ast}\,:\; \pCub(\bSigma, \ast)^d \to \R_{\geq 0}
    \]
    as defined above has the following properties:
    \begin{enumerate}
        \item \( \MVol_{\bSigma, \ast}(z, z, \dots, z) = \MVol_{\bSigma, \ast}(z) \),
        \item \( \MVol_{\bSigma, \ast} \) is symmetric in all arguments,
        \item \( \MVol_{\bSigma, \ast} \) is multilinear with respect to Minkowski addition in each maximal cone.
    \end{enumerate}

    Further, any function \( \pCub(\bSigma, \ast)^d \to \R_{\geq 0} \) satisfying properties 1--3 must be \( \MVol_{\sigma, \ast} \).
\end{proposition}
Our new mixed volume function then is well-defined and is uniquely characterized by the same properties as the original.
Theorem~3.6 of \cite{nowakMixedVolumesNormal2023} extends \th\ref{thm:volDegCorrespondence} to a form that lets us evaluate mixed degrees.
\begin{theorem}\th\label{thm:mixedDeg}
    Let \( \bSigma \subset N \) be a balanced \( d \)-fan.
    Choose an inner product \( \ast \in \inn(N) \) and pseudocubical values \( z_1, \dots, z_d \in \pCub( \bSigma, \ast ) \).
    Then
    \[
        \MVol_{\bSigma, \ast}(z_1, \dots, z_d) = \deg\big( D(z_1) \cdots D(z_d) \big).
    \]
\end{theorem}
This is a successful bridge from the realm of geometry back to algebra.
We are now so close to having all the necessary components to prove our main result.
Not only do we have the link between geometry and algebra, it uses the concept of mixed volumes, which are closely related to log-concave sequences.
However, the Alexandrov-Fenchel inequalities are, classically, very dependent on convexity, and normal complexes are decidedly non-convex.

\subsection{Amazing AF Fans}
Given our extended mixed volume function on normal complexes, the resulting polynomials will sometimes produce coefficients who obey the Alexandrov-Fenchel inequality.
This is in fact a property of the underlying fan itself.
\begin{definition}[Alexandrov-Fenchel Property]\th\label{def:AF}
    Let \( \bSigma \subseteq \R^n \) be a simplicial \( d \)-fan and \( \ast \in \inn(\R^n) \) an inner product.
    We say that \( (\bSigma, \ast) \) is \emph{Alexandrov-Fencel}, or just \emph{AF}, if \( \Cub(\bSigma, \ast) \neq \emptyset \) and
    \[
        \MVol_{\bSigma, \ast}(z_1, z_2, z_3, \dots, z_d)^2 \geq  \MVol_{\bSigma, \ast}(z_1, z_1, z_3, \dots, z_d)\MVol_{\bSigma, \ast}(z_2, z_2, z_3 \dots, z_d)
    \]
    for all \( z_1, z_2, z_3, \dots, z_d \in \Cub(\bSigma, \ast) \).
\end{definition}
If a fan is AF, then taking the mixed volumes of any normal complexes built from it will give rise to a log-concave sequence.
This is naturally weaker than the Alexandrov-Finchel inequality in the convex setting, which is just universally true, but that it is still sometimes true in a non-convex setting is
Our goal now then is to show that Bergman fans of matroids are AF\@.
Luckily, Theorem~5.1 in \cite{nowakMixedVolumesNormal2023} gives us exact criteria to use.

However, to state the theorem, we will need to know how to consider smaller components of a fan, and therefore also normal complexes.
First, we'll consider the neighborhood of a cone in a fan.
\begin{definition}[Neighborhood]\th\label{def:nbr}
    Let \( \bSigma \) be a fan and \( \sigma \in \bSigma \) a cone in the fan.
    The neighborhood of \( \sigma \) in \( \bSigma \) is
    \[
        \nbd(\sigma, \bSigma) = \left\{ \tau \; \middle| \; \tau \preceq \pi, \, \sigma \preceq \pi \right\}.
    \]
\end{definition}
The neighborhood of a cone \( \sigma \) is essentially the collection of all other cones in the fan contain \( \sigma \) as a face.
We then take all faces of those cones to insure that \( \nbd(\sigma, \bSigma) \) is still a fan.
In fact \( \nbd(\sigma, \bSigma) \) is still a fan living in the same space as \( \bSigma \).
If we additionally quotient out the components of \( \sigma \) from the neighborhood we have what is called the star of \( \sigma \).
\begin{definition}[Star]\th\label{def:star}
    Let \( \bSigma \subseteq \R^d\) be a fan and \( \sigma \in \bSigma \) a cone in the fan.
    The \emph{star} of \( \sigma \) is given by
    \[
        \Star(\sigma, \bSigma) = \left\{ \overline{\tau} \; \middle| \; \tau \in \nbd(\sigma, \bSigma) \right\} \subseteq \R^d / \Span(\sigma),
    \]
    where \( \overline{\tau} \) is the equivalence class of \( \tau \) in the quotient space \( \R^d / \Span(\sigma) \).
\end{definition}
When provided with an inner product, we can identify the quotient space \( \R^d / \Span(\sigma)\) with the orthogonal space, \( \Span(\sigma)^\bot \).
Given we are always providing an inner product, we mostly think of the star as the projection of the neighborhood of \( \sigma \) into \( \Span(\sigma)^\bot \).
Since the neighborhood was a fan in the original space, the star will be a fan in this orthogonal space, which we notate as  \( \bSigma^{\sigma} \).

Perhaps more surprisingly, if we have the necessary data to make a normal complex on some fan \( \bSigma \), then for any cone \( \sigma \in \bSigma \), we also have everything we need to define a normal complex on \( \bSigma^\sigma \).
We call these faces of the normal complex, in analogy with polytopes, and the details of their construction and existence can be found in section~4 of \cite{nowakMixedVolumesNormal2023}.
\begin{definition}[Face of Normal Complex]\th\label{def:faceNC}
    Let \( \bSigma \subset \R^n \) be a simplicial \( d \)-fan and \( \ast \in \inn(\R^n) \) be an inner product.
    Given any \( z \in \pCub(\bSigma, \ast) \) and cone \( \sigma \in \Sigma \), the \emph{face} of \( C_{\bSigma, \ast}(z) \) associated to \( \sigma \) is
    \[
        F^\sigma\big(C_{\bSigma, \ast}(z)\big)  = C_{\bSigma^\sigma, \ast^\sigma}(z^\sigma),
    \]
    where \( \ast^\sigma \) is the restriction of \( \ast \) to \( \Span(\sigma)^\bot \) and \( z^\sigma \in \pCub(\bSigma^\sigma, \ast^\sigma) \) is uniquely determined by \( \Sigma, \sigma, \) and \( z \).
\end{definition}
A good thing to note here is there is a dimension reversing relationship between the dimension of \( \sigma \) and \( \bSigma^\tau \).
If \( \bSigma \) is a \( d \)-fan and \( \sigma \) is of dimension \( k \), then \( \bSigma^\sigma \) is pure of dimension \( d-k \).
We would then call \( F^\sigma\big(C_{\bSigma, \ast}(z)\big) \) a \( (d-k) \)-dimensional face of the normal complex \( C_{\bSigma, \ast}(z) \).
This finally gives us enough background information to state Theorem~5.1 of~\cite{nowakMixedVolumesNormal2023}.
\begin{theorem}[Nowak-O-Ross 2022]\th\label{thm:suffAF}
    Let \( \bSigma \) be a balanced \( d \)-fan, and \( \ast \in \Inn(N_\R) \) an inner product such that \( \Cub(\bSigma, \ast) \neq \emptyset \).
    Then \( (\bSigma, \ast) \) is AF if
    \begin{enumerate}[label=\roman*.]
        \item for every cone \( \sigma \in \bSigma(k) \), with \( k \leq d - 2 \),
              \[
                  \Star(\sigma, \bSigma_\cM) \setminus \{ 0 \}
              \]
              is connected and,
        \item for any choice of \( z \in \Cub(\bSigma, \ast) \) the volume of each 2-dimensional face of the associated normal complex, \( C(\bSigma, z) \), is quadratic form whose associated matrix has exactly one positive eigenvalue.
    \end{enumerate}
\end{theorem}
We can provide a little insight into these two criteria.
In the first, connectedness is the classic topological definition of connected, but, for those without a topological background, practically we can think of this as being able draw a path between any two points.
In this case with the origin removed.
The effect of this criteria is that is it tells us our fans can't have ``pinch'' points.
If the cones of the fan share a non-0-dimensional face that's not a facet, then when taking stars within the fan we may get the case that the cones in the star fan meet only at the origin.
Once we then remove the origin, then this will disconnect the star fan, violating the criteria.

For the second condition, first we recall that since we're looking at 2-dimensional faces of normal complexes, these will be polytopal complexes with maximal elements of dimension 2.
The volume of the polytopes will depend on the \( z \)-values that determine the truncation.
Since each polytope's volume will be determined by 2 \( z \)-values each, we can write the volume of the whole complex as a polynomial in \(z_i\) where each monomial is of degree 2.
This is what is called a quadratic form, and there is always a matrix associated to quadratic forms, and it is this matrix's eigenvalues that are of interest to us.

We now finally have laid out every piece of existing work we'll need to prove our main result.
All that's left is to put them together.

\end{document}