\documentclass[12pt,oneside]{../../sfsuthesis}  

\RequirePackage{standalone}
\usepackage[draft]{../../MAThesisOutputFormat}
%====================%
% Packages 
%====================%
%\usepackage{accents}  % Better accents. I'm not using this
\usepackage{enumitem} % Better labels
%\usepackage[explicit]{titlesec}
\usepackage[normalem]{ulem}     % Thing
\usepackage{stackrel} % Stack text nicely
\usepackage{xcolor}   % Nicer Colors

%====================%
% Re-Define 
%====================%
% Sort out the true phi
\def \badphi {\phi}
\def \phi {\varphi}

%====================%
% New Commands 
%====================%
%% Nice Letters
% Blackboard Bold
\newcommand{\R}{\mathbb{R}}
\newcommand{\Z}{\mathbb{Z}}
\newcommand{\bbS}{\mathbb{S}}
% Fancy Math Cal Letters
\newcommand{\I}{\mathcal{I}}
\newcommand{\J}{\mathcal{J}}
\newcommand{\cL}{\mathcal{L}}
\newcommand{\cF}{\mathcal{F}}
% The hipster letters
\newcommand{\sM}{\mathsf{M}}
\newcommand{\bSigma}{\boldsymbol{\Sigma}}
\newcommand{\ow}{\overline{w}}
\newcommand{\oP}{\overline{P}}
\newcommand{\oQ}{\overline{Q}}
\newcommand{\ochi}{\overline{\chi}}
%% Math Operators
\newcommand{\Cub}{\operatorname{Cub}}
\newcommand{\oCub}{\overline{\Cub}}
\newcommand{\Vol}{\operatorname{Vol}}
\newcommand{\oVol}{\overline{\Vol}}
\newcommand{\MVol}{\operatorname{MVol}}
%% Math Symbols
\newcommand{\cl}{\mathrm{cl}}
\newcommand{\rk}{\mathrm{rk}}
\newcommand{\Inn}{\mathrm{Inn}}
\newcommand{\cone}{\mathrm{cone}}

%====================%
% Theorem Environs
%====================%
\newtheorem{dummy}{}[section]
\theoremstyle{definition}
\newtheorem*{Result}{Main Result}

%====================%
% Draft Helpers
%====================%
% \todo : Make a box with todo comments
\newcommand{\todo}[1]{\par \noindent
    \framebox{\begin{minipage}[c]{0.95 \textwidth}
            \textcolor{red}{TO DO:}
            #1 \end{minipage}}\par}
\usepackage[backend=biber,style=numeric]{biblatex}
\addbibresource{../../thesis.bib}

\begin{document}

\chapter{Introduction}

\todo{Add Citations in Introduction}

This thesis will require us to take a tour of mathematics that have been developing for close to a century.
The main result synthesizes modern work, from about ten years right up to last year, about a conjecture, posed in seventies, on a mathematical object first formalized in 1935.
This threads through work of our friends and mentors, Fields Medal winners, and a host of well-known mathematicians from across the last hundred years.
While we think the results alone are quite interesting on their own, much of what has made this project so interesting to us is its broad connections to these various places.
We hope, through the more leisurely pace we are allowed to take in a thesis, to show off this side of the math as well.

\section{What Are We Doing?}

The key players in this work are \textbf{matroids}, a combinatorial object devised to generalize the notion of ``independence''.
Matroids are interesting for a multitude of reasons, but of note to us is that, although they are combinatorial objects, they can be alternatively studied as geometric objects, known as \textbf{Bergmann fans}, and algebraic objects, polynomials rings called \textbf{Chow rings}.
In the early 1970's, a conjecture about the \textbf{characteristic polynomial} of matroids was posed.
The \textbf{Heron-Rota-Welsh conjecture} was, in essence a combinatorial question, and would remain unresolved for almost 50 years.
It was through viewing the problem from the algebro-geometric side of things that Adiprasito, Huh, and Katz were finally able to prove the conjecture true in 2017.
They did this by importing complex machinery from algebraic geometry, known as \emph{Hodge theory}, into the combinatorial world of matroids.
It is an impressive work that, in part, won author June Huh a Fields medal.

In this thesis we wish to offer an alternative proof of the Heron-Rota-Welsh conjecture.
To do this, we too will tackle the problem from a geometric perspective.
In 2021, Nathanson and Ross developed a correspondence between the volume of objects generated from the Bergmann fans of matroids, called \textbf{normal complexes}, and the evaluation of degree maps on the Chow Ring.
This opened the avenue of using the geometric picture of matroids to show characteristics of its algebraic representation.
Using the recent work of Nowak, O\todo{It feels weird to use my name in the 3rd person?}, and Ross, we show this correspondence and certain volumetric properties of normal complexes is sufficient to prove the Heron-Rota-Welsh conjecture.

\section{Why Are We Doing This?}

Where some would ask why, we much prefer to ask ``why not?''.
More seriously, while \textit{a} proof of the Heron-Rolta-Welsh conjecture is not new, having a new viewpoint on something is valuable even just in comparing it to the original.

This is an exciting starting application of the theory of normal complexes.
Compared to combinatorial Hodge theory, normal complexes have a much lower barrier of entry.
That they can prove the same, famously difficult, problem is surprising at least.
We hope to see some of these techniques and tools expanded and applied elsewhere.

\section{Who Is This For?}

By this point we've already introduced quite a few words we don't expect every reader to know offhand.
Our primary goal is that anyone with a few graduate level courses in mathematics under their belt could read this thesis from start to finish and come out with a comprehensive picture of both the setting and the conclusion.
To that end we will be providing context for every word in bold appearing above, linking each of them to the overall picture.

However, we also have some secondary goals in terms of readership.
First, we want this to be of at least some interest to someone already knowledgeable in the field.
While we are confident that any math of real substance in this thesis will be developed elsewhere, if it's going to appear here it might as well at least be useful to a practitioner.
Second, and in somewhat of a contradiction, we want this work to be inviting to a curious non-mathematican.
We believe there is a good opportunity here to allow a layperson to follow along with math they may not be otherwise usually exposed too.

In the true spirit of compromise then, we expect no one to be totally happy with the pacing.
In general, the intention is the complexity of the material will start somewhat low and increase as we go on.
But, there will be technical points interjected in otherwise easy material, and we will attempt to include high level overviews even in sections that really do require a solid mathematical background.
We say this largely to give the reader permission to skip the bits that simply don't interest them.
%However you engage with this thesis, we hope a reader gets something out of it.
%Even better if they were to somewhat enjoy the process.

\end{document}