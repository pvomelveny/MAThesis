\documentclass[12pt,oneside]{../../sfsuthesis} 
 
\RequirePackage{standalone}
\usepackage[draft]{../../MAThesisOutputFormat}
%====================%
% Packages 
%====================%
%\usepackage{accents}  % Better accents. I'm not using this
\usepackage{enumitem} % Better labels
%\usepackage[explicit]{titlesec}
\usepackage[normalem]{ulem}     % Thing
\usepackage{stackrel} % Stack text nicely
\usepackage{xcolor}   % Nicer Colors

%====================%
% Re-Define 
%====================%
% Sort out the true phi
\def \badphi {\phi}
\def \phi {\varphi}

%====================%
% New Commands 
%====================%
%% Nice Letters
% Blackboard Bold
\newcommand{\R}{\mathbb{R}}
\newcommand{\Z}{\mathbb{Z}}
\newcommand{\bbS}{\mathbb{S}}
% Fancy Math Cal Letters
\newcommand{\I}{\mathcal{I}}
\newcommand{\J}{\mathcal{J}}
\newcommand{\cL}{\mathcal{L}}
\newcommand{\cF}{\mathcal{F}}
% The hipster letters
\newcommand{\sM}{\mathsf{M}}
\newcommand{\bSigma}{\boldsymbol{\Sigma}}
\newcommand{\ow}{\overline{w}}
\newcommand{\oP}{\overline{P}}
\newcommand{\oQ}{\overline{Q}}
\newcommand{\ochi}{\overline{\chi}}
%% Math Operators
\newcommand{\Cub}{\operatorname{Cub}}
\newcommand{\oCub}{\overline{\Cub}}
\newcommand{\Vol}{\operatorname{Vol}}
\newcommand{\oVol}{\overline{\Vol}}
\newcommand{\MVol}{\operatorname{MVol}}
%% Math Symbols
\newcommand{\cl}{\mathrm{cl}}
\newcommand{\rk}{\mathrm{rk}}
\newcommand{\Inn}{\mathrm{Inn}}
\newcommand{\cone}{\mathrm{cone}}

%====================%
% Theorem Environs
%====================%
\newtheorem{dummy}{}[section]
\theoremstyle{definition}
\newtheorem*{Result}{Main Result}

%====================%
% Draft Helpers
%====================%
% \todo : Make a box with todo comments
\newcommand{\todo}[1]{\par \noindent
    \framebox{\begin{minipage}[c]{0.95 \textwidth}
            \textcolor{red}{TO DO:}
            #1 \end{minipage}}\par}
\usepackage[backend=biber,style=numeric]{biblatex}
\addbibresource{../../thesis.bib}

\begin{document}

\chapter{Chow Rings and Bergmann Fans}

\section{Chow Rings}
\todo{Introduce the Chow Ring of a matroid. From lattice of flats to quotient ring}
\todo{Use our example matroid and construct its Chow Ring}

\subsection{Properties of Chow Rings}
\todo{Use words like \emph{homogeneous polynomial}, \emph{graded ring}, etc\dots}

\subsection{The Degree Map}
\todo{How do I explain this?
    I guess I can at least say it's linear and sends terms of full degree to 1.
    Maybe I'll understand it this time around}

\section{Bergmann Fans}
\todo{Quick definition of a fan?}

\subsection{How to make a Bergmann Fan}

\todo{Show definition from Chow Ring to Bergmann Fan}
\todo{Work our small example into a fan}

\subsection{Properties of Bergmann Fans}

\todo{Figure out their important properties. They're unimodal, so we'll include that.
    Oh, they're balanced as well (and so tropical). What am I missing?}
\todo{Define star of a fan. Reference A-H-K; star of a Bergmann fan is again a Bergmann fan? Or something like that}

\section{Relationship with the Characteristic Polynomial}

\todo{Come up with a nice way of relating the reduced characteristic polynomial with our ring (and therefore fan)}
\todo{Define \(\alpha\) and \(\beta\). Here or in a subsection? Or should it be up when we introduce the ring itself?}

\subsection{How to show Log-Concavity}
\todo{Offer a self-contained proof that log-concavity is equivalent to showing that one particular relationship between \(\alpha\) and \(\beta\) found in A-H-K.
    They have the proof, but it requires digging through citations.
    Should be able to consolidate it.}

\end{document}